\chapter{Introduction}
\label{intro}

%% TODO: improve the feature sheet with a nice table to highlight new features.

\quoting{I just tried out ejabberd and was impressed both by ejabberd itself and the language it is written in, Erlang. ---
Joeri} 

%ejabberd is a free and open source instant messaging server written in Erlang. ejabberd is cross-platform, distributed, fault-tolerant, and based on open standards to achieve real-time communication (Jabber/XMPP).

\ejabberd{} is a \marking{free and open source} instant messaging server written in \footahref{http://www.erlang.org/}{Erlang}.

\ejabberd{} is \marking{cross-platform}, distributed, fault-tolerant, and based on open standards to achieve real-time communication.

\ejabberd{} is designed to be a \marking{rock-solid and feature rich} XMPP server.

\ejabberd{} is suitable for small deployments, whether they need to be \marking{scalable} or not, as well as extremely big deployments.

%\subsection{Layout with example deployment (title needs a better name)}
%\label{layout}

%In this section there will be a graphical overview like these:\\
%\verb|http://www.tipic.com/var/timp/timp_dep.gif| \\
%\verb|http://www.jabber.com/images/jabber_Com_Platform.jpg| \\
%\verb|http://www.antepo.com/files/OPN45systemdatasheet.pdf| \\

%A page full with names of Jabber client that are known to work with ejabberd. \begin{tiny}tiny font\end{tiny}

%\subsection{Try It Today}
%\label{trytoday}

%(Not sure if I will include/finish this section for the next version.)

%\begin{itemize}
%\item Erlang REPOS
%\item Packages in distributions
%\item Windows binary
%\item source tar.gz
%\item Migration from Jabberd14 (and so also Jabberd2 because you can migrate from version 2 back to 14) and Jabber Inc. XCP possible.
%\end{itemize}

\newpage
\section{Key Features}
\label{keyfeatures}
\ind{features!key features}

\quoting{Erlang seems to be tailor-made for writing stable, robust servers. ---
Peter Saint-Andr\'e, Executive Director of the Jabber Software Foundation}

\ejabberd{} is:
\begin{itemize}
\item \marking{Cross-platform:} \ejabberd{} runs under Microsoft Windows and Unix derived systems such as Linux, FreeBSD and NetBSD.

\item \marking{Distributed:} You can run \ejabberd{} on a cluster of machines and all of them will serve the same \Jabber{} domain(s). When you need more capacity you can simply add a new cheap node to your cluster. Accordingly, you do not need to buy an expensive high-end machine to support tens of thousands concurrent users.

\item \marking{Fault-tolerant:} You can deploy an \ejabberd{} cluster so that all the information required for a properly working service will be replicated permanently on all nodes. This means that if one of the nodes crashes, the others will continue working without disruption. In addition, nodes also can be added or replaced `on the fly'.

\item \marking{Administrator Friendly:} \ejabberd{} is built on top of the Open Source Erlang. As a result you do not need to install an external database, an external web server, amongst others because everything is already included, and ready to run out of the box. Other administrator benefits include:
\begin{itemize}
\item Comprehensive documentation.
\item Straightforward installers for Linux, Mac OS X, and Windows. %%\improved{}
\item Web Administration.
\item Shared Roster Groups.
\item Command line administration tool. %%\improved{}
\item Can integrate with existing authentication mechanisms.
\item Capability to send announce messages.
\end{itemize}

\item \marking{Internationalized:} \ejabberd{} leads in internationalization. Hence it is very well suited in a globalized world. Related features are:
\begin{itemize}
\item Translated in 20 languages. %%\improved{}
\item Support for \footahref{http://www.ietf.org/rfc/rfc3490.txt}{IDNA}.
\end{itemize}

\item \marking{Open Standards:} \ejabberd{} is the first Open Source Jabber server claiming to fully comply to the XMPP standard.
\begin{itemize}
\item Fully XMPP compliant.
\item XML-based protocol.
\item \footahref{http://www.ejabberd.im/protocols}{Many protocols supported}.
\end{itemize}

\end{itemize}

\newpage

\section{Additional Features}
\label{addfeatures}
\ind{features!additional features}

\quoting{ejabberd is making inroads to solving the "buggy incomplete server" problem ---
Justin Karneges, Founder of the Psi and the Delta projects}

Moreover, \ejabberd{} comes with a wide range of other state-of-the-art features:
\begin{itemize}
\item Modular
\begin{itemize}
\item Load only the modules you want.
\item Extend \ejabberd{} with your own custom modules.
\end{itemize}
\item Security
\begin{itemize}
\item SASL and STARTTLS for c2s and s2s connections.
\item STARTTLS and Dialback s2s connections.
\item Web Admin accessible via HTTPS secure access.
\end{itemize}
\item Databases
\begin{itemize}
\item Internal database for fast deployment (Mnesia).
\item Native MySQL support.
\item Native PostgreSQL support.
\item ODBC data storage support.
\item Microsoft SQL Server support. %%\new{}
\end{itemize}
\item Authentication
\begin{itemize}
\item Internal Authentication.
\item PAM, LDAP and ODBC.  %%\improved{}
\item External Authentication script.
\end{itemize}
\item Others
\begin{itemize}
\item Support for virtual hosting.
\item Compressing XML streams with Stream Compression (\xepref{0138}).
\item Statistics via Statistics Gathering (\xepref{0039}).
\item IPv6 support both for c2s and s2s connections.
\item \txepref{0045}{Multi-User Chat} module with support for clustering and HTML logging. %%\improved{}
\item Users Directory based on users vCards.
\item \txepref{0060}{Publish-Subscribe} component with support for \txepref{00163}{Personal Eventing via Pubsub}.
\item Support for web clients: \txepref{0025}{HTTP Polling} and \txepref{0206}{HTTP Binding (BOSH)} services.
\item IRC transport.
\item Component support: interface with networks such as AIM, ICQ and MSN installing special tranports.
\end{itemize}
\end{itemize}
