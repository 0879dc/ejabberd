\documentclass[a4paper,10pt]{article}

\usepackage{graphics}
\usepackage{hevea}
\usepackage{verbatim}

\usepackage[twosideshift=0pt]{geometry}

\usepackage[pdftex,colorlinks,unicode,urlcolor=blue,linkcolor=blue,pdftitle=Ejabberd\
        Installation\ and\ Operation\ Guide,pdfauthor=Alexey\
        Shchepin,pdfsubject=ejabberd,pdfkeywords=ejabberd]{hyperref}

\newcommand{\logoscale}{0.7}
\newcommand{\imgscale}{0.58}
\newcommand{\insimg}[1]{\insscaleimg{\imgscale}{#1}}

\newcommand{\insscaleimg}[2]{
  \imgsrc{#2}{}
  \begin{latexonly}
    \scalebox{#1}{\includegraphics{#2}}
  \end{latexonly}
}

\newcommand{\bracehack}{\def\{{\char"7B}\def\}{\char"7D}}

\newcommand{\ns}[1]{\texttt{#1}}
\newcommand{\jid}[1]{\texttt{#1}}
\newcommand{\option}[1]{\texttt{#1}}
\newcommand{\poption}[1]{{\bracehack\texttt{#1}}}
\newcommand{\node}[1]{\texttt{#1}}
\newcommand{\term}[1]{\texttt{#1}}
\newcommand{\shell}[1]{\texttt{#1}}
\newcommand{\ejabberd}{\texttt{ejabberd}}
\newcommand{\Jabber}{Jabber}

\newcommand{\module}[1]{\texttt{#1}}
\newcommand{\modannounce}{\module{mod\_announce}}
\newcommand{\modconfigure}{\module{mod\_configure}}
\newcommand{\moddisco}{\module{mod\_disco}}
\newcommand{\modirc}{\module{mod\_irc}}
\newcommand{\modlast}{\module{mod\_last}}
\newcommand{\modmuc}{\module{mod\_muc}}
\newcommand{\modecho}{\module{mod\_echo}}
\newcommand{\modoffline}{\module{mod\_offline}}
\newcommand{\modprivacy}{\module{mod\_privacy}}
\newcommand{\modprivate}{\module{mod\_private}}
\newcommand{\modpubsub}{\module{mod\_pubsub}}
\newcommand{\modregister}{\module{mod\_register}}
\newcommand{\modroster}{\module{mod\_roster}}
\newcommand{\modservicelog}{\module{mod\_service\_log}}
\newcommand{\modsharedroster}{\module{mod\_shared\_roster}}
\newcommand{\modstats}{\module{mod\_stats}}
\newcommand{\modtime}{\module{mod\_time}}
\newcommand{\modvcard}{\module{mod\_vcard}}
\newcommand{\modversion}{\module{mod\_version}}

\newcommand{\titem}[1]{\item[\bracehack\texttt{#1}]}

%\setcounter{tocdepth}{3}
\begin{latexonly}
\global\parskip=9pt plus 3pt minus 1pt
\global\parindent=0pt

\gdef\ahrefurl#1{\href{#1}{\texttt{#1}}}
\gdef\footahref#1#2{#2\footnote{\href{#1}{\texttt{#1}}}}
\end{latexonly}

\newcommand{\tjepref}[2]{\footahref{http://www.jabber.org/jeps/jep-#1.html}{#2}}
\newcommand{\jepref}[1]{\tjepref{#1}{JEP-#1}}

\newcommand{\iqdiscitem}[1]{\titem{iqdisc} #1 IQ queries processing
discipline (see~\ref{sec:modiqdiscoption}).}
\newcommand{\hostitem}[1]{
  \titem{host} Defines hostname of the service
  (see~\ref{sec:modhostoption}).
  \titem{hosts} Defines hostnames of the service
  (see~\ref{sec:modhostsoption}).  If neither \texttt{host} nor \texttt{hosts}
  are not present, then prefix \jid{#1.} is added to all \ejabberd{} hostnames.
}

\title{Ejabberd Installation and Operation Guide}
\author{Alexey Shchepin \\
  \ahrefurl{mailto:alexey@sevcom.net} \\
  \ahrefurl{xmpp:aleksey@jabber.ru}}
\date{April 18, 2005}

\begin{document}
\begin{titlepage}
  \maketitle{}
  
  {\centering
    \insscaleimg{\logoscale}{logo.png}
    \par
  }
\end{titlepage}
%\newpage
\tableofcontents{}

\newpage
\section{Introduction}
\label{sec:intro}

\ejabberd{} is a Free and Open Source fault-tolerant distributed \Jabber{}
server.  It is written mostly in Erlang.

The main features of \ejabberd{} are:
\begin{itemize}
\item Works on most of popular platforms: *nix (tested on Linux, FreeBSD and
  NetBSD) and Win32
\item Distributed: You can run \ejabberd{} on a cluster of machines to let all of
  them serve one Jabber domain.
\item Fault-tolerance: You can setup an \ejabberd{} cluster so that all the
  information required for a properly working service will be stored
  permanently on more than one node.  This means that if one of the nodes
  crashes, then the others will continue working without disruption.
  You can also add or replace nodes ``on the fly''.
\item Support for virtual hosting
\item Built-in \tjepref{0045}{Multi-User Chat} service
\item Built-in IRC transport
\item Built-in \tjepref{0060}{Publish-Subscribe} service
\item Built-in Jabber Users Directory service based on users vCards
\item Built-in web-based administration interface
\item Built-in \tjepref{0025}{HTTP Polling} service
\item SSL support
\item Support for LDAP authentication
\item Ability to interface with external components (JIT, MSN-t, Yahoo-t, etc.)
\item Migration from jabberd14 is possible
\item Mostly XMPP-compliant
\item Support for \tjepref{0030}{Service Discovery}.
\item Support for \tjepref{0039}{Statistics Gathering}.
\item Support for \ns{xml:lang}
\end{itemize}

The misfeatures of \ejabberd{} are:
\begin{itemize}
\item No support for authentication and STARTTLS in S2S connections
\item Access rules can be defined only for global conext, not for specific
  virtual host
\end{itemize}


\section{Installation from Source}
\label{sec:installation}

\subsection{Installation Requirements}
\label{sec:installreq}

\subsubsection{Unix}
\label{sec:installrequnix}

To compile \ejabberd{}, you will need the following packages:
\begin{itemize}
\item GNU Make;
\item GCC;
\item libexpat 1.95 or later;
\item Erlang/OTP R8B or later;
\item OpenSSL 0.9.6 or later (optional).
\end{itemize}

\subsubsection{Windows}
\label{sec:installreqwin}

To compile \ejabberd{} in MS Windows environment, you will need the following
packages:
\begin{itemize}
\item MS Visual C++ 6.0 Compiler
\item \footahref{http://erlang.org/download/otp\_win32\_R10B-1a.exe}{Erlang/OTP R10B-1a}
\item \footahref{http://prdownloads.sourceforge.net/expat/expat\_win32bin\_1\_95\_7.exe?download}{Expat 1.95.7}
\item
\footahref{http://ftp.gnu.org/pub/gnu/libiconv/libiconv-1.9.1.tar.gz}{Iconv 1.9.1}
(optional)
\item \footahref{http://www.slproweb.com/products/Win32OpenSSL.html}{Shining Light OpenSSL}
(to enable SSL connections)
\end{itemize}


\subsection{Obtaining}
\label{sec:obtaining}

Stable \ejabberd{} release can be obtained at
\ahrefurl{http://www.process-one.net/en/projects/ejabberd/download.html}.

The latest alpha version can be retrieved from Subversion repository\@.
\begin{verbatim}
  svn co svn://svn.process-one.net/opt/data/svn/ejabberd/trunk ejabberd
\end{verbatim}


\subsection{Compilation}
\label{sec:compilation}
\subsubsection{Unix}
\label{sec:compilationunix}

\begin{verbatim}
  ./configure
  make
  su
  make install
\end{verbatim}

This will install \ejabberd{} to \verb|/var/lib/ejabberd| directory,
\verb|ejabberd.cfg| to \verb|/etc/ejabberd| directory and create
\verb|/var/log/ejabberd| directory for log files.

\subsubsection{Windows}
\label{sec:compilationwin}

\begin{itemize}
\item Install Erlang emulator (for example, into \verb|C:\Program Files\erl5.3|).
\item Install Expat library into \verb|C:\Program Files\Expat-1.95.7|
  directory.
  
  Copy file \verb|C:\Program Files\Expat-1.95.7\Libs\libexpat.dll|
  to your Windows system directory (for example, \verb|C:\WINNT| or
  \verb|C:\WINNT\System32|)
\item Build and install Iconv library into \verb|C:\Program Files\iconv-1.9.1| directory.
  
  Copy file \verb|C:\Program Files\iconv-1.9.1\bin\iconv.dll| to your
  Windows system directory.
  
  Note: Instead of copying libexpat.dll and iconv.dll to Windows
  directory, you can add directories
  \verb|C:\Program Files\Expat-1.95.7\Libs| and
  \verb|C:\Program Files\iconv-1.9.1\bin| to \verb|PATH| environment
  variable.
\item Being in \verb|ejabberd\src| directory run:
\begin{verbatim}
configure.bat
nmake -f Makefile.win32
\end{verbatim}
\item Edit file \verb|ejabberd\src\ejabberd.cfg| and run
\begin{verbatim}
werl -s ejabberd -name ejabberd
\end{verbatim}
\end{itemize}

%\subsection{Initial Configuration}
%\label{sec:initconfig}


\subsection{Starting}
\label{sec:starting}

To start \ejabberd{}, use the following command:
\begin{verbatim}
  erl -pa /var/lib/ejabberd/ebin -name ejabberd -s ejabberd
\end{verbatim}
or
\begin{verbatim}
  erl -pa /var/lib/ejabberd/ebin -sname ejabberd -s ejabberd
\end{verbatim}
In the latter case Erlang node will be identified using only first part of host
name, i.\,e. other Erlang nodes outside this domain can't contact this node.

Note that when using above command \ejabberd{} will search for config file
in current directory and will use current directory for storing user database
and logging.

To specify path to config file, log files and Mnesia database directory,
you may use the following command:
\begin{verbatim}
  erl -pa /var/lib/ejabberd/ebin \
      -sname ejabberd \
      -s ejabberd \
      -ejabberd config \"/etc/ejabberd/ejabberd.cfg\" \
                log_path \"/var/log/ejabberd/ejabberd.log\" \
      -sasl sasl_error_logger \{file,\"/var/log/ejabberd/sasl.log\"\} \
      -mnesia dir \"/var/lib/ejabberd/spool\"
\end{verbatim}

You can find other useful options in Erlang manual page (\shell{erl -man erl}).

To use more than 1024 connections, you should set environment variable
\verb|ERL_MAX_PORTS|:
\begin{verbatim}
  export ERL_MAX_PORTS=32000
\end{verbatim}
Note that with this value \ejabberd{} will use more memory (approximately 6MB
more).

To reduce memory usage, you may set environment variable
\verb|ERL_FULLSWEEP_AFTER|:
\begin{verbatim}
  export ERL_FULLSWEEP_AFTER=0
\end{verbatim}
But in this case \ejabberd{} can start to work slower.


\section{Configuration}
\label{sec:configuration}

\subsection{Initial Configuration}
\label{sec:initconfig}

The configuration file is initially loaded the first time \ejabberd{} is
executed, when it is parsed and stored in a database.  Subsequently the
configuration is loaded from the database and any commands in the configuration
file are appended to the entries in the database.  The configuration file
consists of a sequence of Erlang terms. Parts of lines after \term{`\%'} sign
are ignored.  Each term is tuple, where first element is name of option, and
other are option values. E.\,g.\ if this file does not contain a ``host''
definition, then old value stored in the database will be used.


To override old values stored in the database the following lines can be added
in config:
\begin{verbatim}
  override_global.
  override_local.
  override_acls.
\end{verbatim}
With this lines old global or local options or ACLs will be removed before
adding new ones.


\subsubsection{Host Names}
\label{sec:confighostname}

Option \option{hosts} defines a list of \Jabber{} domains that \ejabberd{}
serves.  E.\,g.\ to serve \jid{example.org} and \jid{example.com} domains add
the following line in the config:
\begin{verbatim}
  {hosts, ["example.org", "example.com"]}.
\end{verbatim}

Option \option{host} defines one \Jabber{} domain that \ejabberd{} serves.
E.\,g.\ to serve only \jid{example.org} domain add the following line in the
config:
\begin{verbatim}
  {host, "example.org"}.
\end{verbatim}
It have the same effect as
\begin{verbatim}
  {hosts, ["example.org"]}.
\end{verbatim}

%This option is mandatory.

\subsubsection{Default Language}
\label{sec:configlanguage}

Option \option{language} defines default language of \ejabberd{} messages, sent
to users. Default value is \term{"en"}. In order to take effect there must be a
translation file \term{<language>.msg} in \ejabberd{} \term{msgs} directory.
E.\,g.\ to use Russian as default language add the following line in the config:
\begin{verbatim}
  {language, "ru"}.
\end{verbatim}


\subsubsection{Access Rules}
\label{sec:configaccess}

Access control in \ejabberd{} is performed via Access Control Lists (ACL).  The
declarations of ACL in config file have following syntax:
\begin{verbatim}
  {acl, <aclname>, {<acltype>, ...}}.
\end{verbatim}

\term{<acltype>} can be one of following:
\begin{description}
\titem{all} Matches all JIDs.  Example:
\begin{verbatim}
{acl, all, all}.
\end{verbatim}
\titem{\{user, <username>\}} Matches user with name
  \term{<username>} at the first virtual host.  Example:
\begin{verbatim}
{acl, admin, {user, "aleksey"}}.
\end{verbatim}
\titem{\{user, <username>, <server>\}} Matches user with JID
  \term{<username>@<server>} and any resource.  Example:
\begin{verbatim}
{acl, admin, {user, "aleksey", "jabber.ru"}}.
\end{verbatim}
\titem{\{server, <server>\}} Matches any JID from server
  \term{<server>}.  Example:
\begin{verbatim}
{acl, jabberorg, {server, "jabber.org"}}.
\end{verbatim}
\titem{\{user\_regexp, <regexp>\}} Matches local user with name that
  matches \term{<regexp>} at the first virtual host.  Example:
\begin{verbatim}
{acl, tests, {user, "^test[0-9]*$"}}.
\end{verbatim}
%$
\titem{\{user\_regexp, <regexp>, <server>\}} Matches user with name
  that matches \term{<regexp>} and from server \term{<server>}.  Example:
\begin{verbatim}
{acl, tests, {user, "^test", "example.org"}}.
\end{verbatim}
\titem{\{server\_regexp, <regexp>\}} Matches any JID from server that
  matches \term{<regexp>}.  Example:
\begin{verbatim}
{acl, icq, {server, "^icq\\."}}.
\end{verbatim}
\titem{\{node\_regexp, <user\_regexp>, <server\_regexp>\}} Matches user
  with name that matches \term{<user\_regexp>} and from server that matches
  \term{<server\_regexp>}.  Example:
\begin{verbatim}
{acl, aleksey, {node_regexp, "^aleksey$", "^jabber.(ru|org)$"}}.
\end{verbatim}
\titem{\{user\_glob, <glob>\}}
\titem{\{user\_glob, <glob>, <server>\}}
\titem{\{server\_glob, <glob>\}}
\titem{\{node\_glob, <user\_glob>, <server\_glob>\}} This is same as
  above, but uses shell glob patterns instead of regexp.  These patterns can
  have following special characters:
  \begin{description}
  \titem{*} matches any string including the null string.
  \titem{?} matches any single character.
  \titem{[...]} matches any of the enclosed characters.  Character
    ranges are specified by a pair of characters separated by a \term{`-'}.
    If the first character after \term{`['} is a \term{`!'}, then any
    character not enclosed is matched.
  \end{description}
\end{description}

The following ACLs are pre-defined:
\begin{description}
\titem{all} Matches all JIDs.
\titem{none} Matches none JIDs.
\end{description}

An entry allowing or denying access to different services would look similar to
this:
\begin{verbatim}
  {access, <accessname>, [{allow, <aclname>},
                          {deny, <aclname>},
                          ...
                         ]}.
\end{verbatim}
When a JID is checked to have access to \term{<accessname>}, the server
sequentially checks if this JID mathes one of the ACLs that are second elements
in each tuple in list.  If it is matched, then the first element of matched
tuple is returned else ``\term{deny}'' is returned.

Example:
\begin{verbatim}
  {access, configure, [{allow, admin}]}.
  {access, something, [{deny, badmans},
                       {allow, all}]}.
\end{verbatim}

Following access rules pre-defined:
\begin{description}
\titem{all} Always returns ``\term{allow}''
\titem{none} Always returns ``\term{deny}''
\end{description}


\subsubsection{Shapers Configuration}
\label{sec:configshaper}

With shapers is possible to bound connection traffic.  The declarations of
shapers in config file have following syntax:
\begin{verbatim}
  {shaper, <shapername>, <kind>}.
\end{verbatim}
Currently implemented only one kind of shaper: \term{maxrate}.  It have
following syntax:
\begin{verbatim}
  {maxrate, <rate>}
\end{verbatim}
where \term{<rate>} means maximum allowed incomig rate in bytes/second.
E.\,g.\ to define shaper with name ``\term{normal}'' and maximum allowed rate
1000\,bytes/s, add following line in config:
\begin{verbatim}
  {shaper, normal, {maxrate, 1000}}.
\end{verbatim}


\subsubsection{Listened Sockets}
\label{sec:configlistened}

Option \option{listen} defines list of listened sockets and what services
runned on them.  Each element of list is a tuple with following elements:
\begin{itemize}
\item Port number;
\item Module that serves this port;
\item Options to this module.
\end{itemize}

Currently these modules are implemented:
\begin{description}
  \titem{ejabberd\_c2s} This module serves C2S connections.
  
  The following options are defined:
  \begin{description}
    \titem{\{access, <access rule>\}} This option defines access of users
    to this C2S port.  Default value is ``\term{all}''.
    \titem{\{shaper, <access rule>\}} This option is like previous, but
    use shapers instead of ``\term{allow}'' and ``\term{deny}''.  Default
    value is ``\term{none}''.
    \titem{\{ip, IPAddress\}} This option specifies which network interface to
    listen on. For example \verb|{ip, {192, 168, 1, 1}}|.
    \titem{inet6} Set up the socket for IPv6.
    \titem{starttls} This option specifies that STARTTLS extension is available
    on connections to this port.  You should also set ``\verb|certfile|''
    option.
    \titem{tls} This option specifies that traffic on this port will be
    encrypted using SSL immediately after connecting.  You should also set
    ``\verb|certfile|'' option.
    \titem{ssl} This option specifies that traffic on this port will be
    encrypted using SSL.  You should also set ``\verb|certfile|'' option.  It
    is recommended to use \term{tls} option instead.
    \titem{\{certfile, Path\}} Path to a file containing the SSL certificate.
  \end{description}
  \titem{ejabberd\_s2s\_in} This module serves incoming S2S connections.
  \titem{ejabberd\_service} This module serves connections from \Jabber{}
  services (i.\,e.\ that use the \ns{jabber:component:accept} namespace).
  
  The following additional options are defined for \term{ejabberd\_service}
  (options \option{access}, \option{shaper}, \option{ip}, \option{inet6} are
  still valid):
  \begin{description}
    \titem{\{host, Hostname, [HostOptions]\}} This option defines hostname of connected
    service and allows to specify additional options, e.\,g.\
    \poption{\{password, Secret\}}.
    \titem{\{hosts, [Hostnames], [HostOptions]\}} The same as above, but allows to
    specify several hostnames.
  \end{description}
  \titem{ejabberd\_http} This module serves incoming HTTP connections.

  The following options are defined:
  \begin{description}
    \titem{http\_poll} This option enables \jepref{0025} (HTTP Polling)
    support.  It is available then at \verb|http://server:port/http-poll/|.
    
    \titem{web\_admin} This option enables web-based interface for \ejabberd{}
    administration which is available at \verb|http://server:port/admin/|,
    login and password should be equal to username and password of one of
    registered users who have permission defined in ``configure'' access rule.
  \end{description}
\end{description}

For example, the following configuration defines that:
\begin{itemize}
\item C2S connections are listened on port 5222 and 5223 (SSL) and denied for
  user ``\term{bad}''
\item S2S connections are listened on port 5269
\item HTTP connections are listened on port 5280 and administration interface
  and HTTP Polling support are enabled
\item All users except admins have traffic limit 1000\,B/s
\item AIM transport \jid{aim.example.org} is connected to port 5233 with
  password ``\term{aimsecret}''
\item JIT transports \jid{icq.example.org} and \jid{sms.example.org} are
  connected to port 5234 with password ``\term{jitsecret}''
\item MSN transport \jid{msn.example.org} is connected to port 5235 with
  password ``\term{msnsecret}''
\item Yahoo! transport \jid{yahoo.example.org} is connected to port 5236 with
  password ``\term{yahoosecret}''
\item Gadu-Gadu transport \jid{gg.example.org} is connected to port 5237 with
  password ``\term{ggsecret}''
\item ILE service \jid{ile.example.org} is connected to port 5238 with
  password ``\term{ilesecret}''
\end{itemize}
\begin{verbatim}
  {acl, blocked, {user, "bad"}}.
  {access, c2s, [{deny, blocked},
                 {allow, all}]}.
  {shaper, normal, {maxrate, 1000}}.
  {access, c2s_shaper, [{none, admin},
                        {normal, all}]}.
  {listen,
   [{5222, ejabberd_c2s,     [{access, c2s}, {shaper, c2s_shaper}]},
    {5223, ejabberd_c2s,     [{access, c2s},
                              ssl, {certfile, "/path/to/ssl.pem"}]},
    {5269, ejabberd_s2s_in,  []},
    {5280, ejabberd_http,    [http_poll, web_admin]},
    {5233, ejabberd_service, [{host, "aim.example.org",
                               [{password, "aimsecret"}]}]},
    {5234, ejabberd_service, [{hosts, ["icq.example.org", "sms.example.org"],
                               [{password, "jitsecret"}]}]},
    {5235, ejabberd_service, [{host, "msn.example.org",
                               [{password, "msnsecret"}]}]},
    {5236, ejabberd_service, [{host, "yahoo.example.org",
                               [{password, "yahoosecret"}]}]},
    {5237, ejabberd_service, [{host, "gg.example.org",
                               [{password, "ggsecret"}]}]},
    {5238, ejabberd_service, [{host, "ile.example.org",
                               [{password, "ilesecret"}]}]}
   ]
  }.
\end{verbatim}
Note, that for jabberd14- or wpjabberd-based services you have to make the
transports log and do XDB by themselves:
\begin{verbatim}
  <!--
     You have to add elogger and rlogger entries here when using ejabberd.
     In this case the transport will do the logging.
  -->

  <log id='logger'>
    <host/>
    <logtype/>
    <format>%d: [%t] (%h): %s</format>
    <file>/var/log/jabber/service.log</file>
  </log>

  <!--
     Some Jabber server implementations do not provide
     XDB services (for example jabberd 2.0 and ejabberd).
     xdb_file_so is loaded in to handle all XDB requests.
  -->

  <xdb id="xdb">
    <host/>
    <load>
      <!-- this is a lib of wpjabber or jabberd -->
      <xdb_file>/usr/lib/jabber/xdb_file.so</xdb_file>
      </load>
    <xdb_file xmlns="jabber:config:xdb_file">
      <spool><jabberd:cmdline flag='s'>/var/spool/jabber</jabberd:cmdline></spool>
    </xdb_file>
  </xdb>
\end{verbatim}


\subsubsection{Modules}
\label{sec:configmodules}

Option \term{modules} defines the list of modules that will be loaded after
\ejabberd{} startup.  Each list element is a tuple where first element is a
name of a module and second is list of options to this module.  See
section~\ref{sec:modules} for detailed information on each module.

Example:
\begin{verbatim}
  {modules,
   [{mod_register,  []},
    {mod_roster,    []},
    {mod_privacy,   []},
    {mod_configure, []},
    {mod_disco,     []},
    {mod_stats,     []},
    {mod_vcard,     []},
    {mod_offline,   []},
    {mod_announce,  [{access, announce}]},
    {mod_echo,      [{host, "echo.example.org"}]},
    {mod_private,   []},
    {mod_irc,       []},
    {mod_muc,       []},
    {mod_pubsub,    []},
    {mod_time,      [{iqdisc, no_queue}]},
    {mod_last,      []},
    {mod_version,   []}
   ]}.
\end{verbatim}


\subsection{Online Configuration and Monitoring}
\label{sec:onlineconfig}

\subsubsection{Web-based Administration Interface}
\label{sec:webadm}

To perform online reconfiguration of \ejabberd{} you need to enable
\term{ejabberd\_http} listener with option \term{web\_admin} (see
section~\ref{sec:configlistened}).  After that you can open URL
\verb|http://server:port/admin/| with you favorite web-browser and enter
username and password of an \ejabberd{} user with administrator rights.  E.\,g.
with such config:
\begin{verbatim}
  ...
  {host, "example.org"}.
  ...
  {listen,
   [...
    {5280, ejabberd_http, [web_admin]},
    ...
   ]
  }.
\end{verbatim}
you should enter URL \verb|http://example.org:5280/admin/|.  After
authentication you should see something like in figure~\ref{fig:webadmmain}.
\begin{figure}[htbp]
  \centering
  \insimg{webadmmain.png}
  \caption{Web-administration top page}
  \label{fig:webadmmain}
\end{figure}
Here you can edit access restrictions, manage users, create backup files,
manage DB, enable/disable listened ports, and view statistics.


\subsubsection{\term{ejabberdctl} tool}
\label{sec:ejabberdctl}

It is possible to do some administration operations using \term{ejabberdctl}
command-line tool.  You can check available options running this command
without arguments:
\begin{verbatim}
% ejabberdctl
Usage: ejabberdctl node command

Available commands:
  stop                          stop ejabberd
  restart                       restart ejabberd
  reopen-log                    reopen log file
  register user password        register a user
  unregister user               unregister a user
  backup file                   store a database backup in file
  restore file                  restore a database backup from file
  install-fallback file         install a database fallback from file
  dump file                     dump a database in a text file
  load file                     restore a database from a text file
  registered-users              list all registered users

Example:
  ejabberdctl ejabberd@host restart
\end{verbatim}



\section{Clustering}
\label{sec:clustering}


\subsection{How it works}
\label{sec:howitworks}

A \Jabber{} domain is served by one or more \ejabberd{} nodes.  These nodes can
be runned on different machines that are connected via a network.  They all
must have the ability to connect to port 4369 of all another nodes, and must
have the same magic cookie (see Erlang/OTP documentation, in other words the
file \term{\~{}ejabberd/.erlang.cookie} must be the same on all nodes). This is
needed because all nodes exchange information about connected users, S2S
connections, registered services, etc\ldots

Each \ejabberd{} node have following modules:
\begin{itemize}
\item router;
\item local router.
\item session manager;
\item S2S manager;
\end{itemize}


\subsubsection{Router}

This module is the main router of \Jabber{} packets on each node.  It
routes them based on their destinations domains.  It uses a global
routing table.  A domain of packet destination is searched in the
routing table, and if it is found, then the packet is routed to
appropriate process.  If no, then it is sent to the S2S manager.


\subsubsection{Local Router}

This module routes packets which have a destination domain equal to
this server name.  If destination JID has a non-empty user part, then
it is routed to the session manager, else it is processed depending on
its content.


\subsubsection{Session Manager}

This module routes packets to local users.  It searches to what user
resource a packet must be sent via a presence table.  Then packet is
either routed to appropriate C2S process, or stored in offline
storage, or bounced back.


\subsubsection{S2S Manager}

This module routes packets to other \Jabber{} servers.  First, it
checks if an opened S2S connection from the domain of the packet
source to the domain of packet destination is existing.  If it is
existing, then the S2S manager routes the packet to the process
serving this connection, else a new connection is opened.


\subsection{How to setup ejabberd cluster}
\label{sec:cluster}

Suppose you already setuped ejabberd on one of machines (\term{first}), and
you need to setup another one to make \ejabberd{} cluster.  Then do
following steps:

\begin{enumerate}
\item Copy \verb|~ejabberd/.erlang.cookie| file from \term{first} to
  \term{second}.
  
  (alt) You can also add ``\verb|-cookie content_of_.erlang.cookie|''
  option to all ``\shell{erl}'' commands below.
  
\item On \term{second} run under `\term{ejabberd}' user in a directory
  where ejabberd will work later the following command:

\begin{verbatim}
erl -sname ejabberd \
    -mnesia extra_db_nodes "['ejabberd@first']" \
    -s mnesia
\end{verbatim}
  
  This will start mnesia serving same DB as \node{ejabberd@first}.
  You can check this running ``\verb|mnesia:info().|'' command.  You
  should see a lot of remote tables and a line like the following:

\begin{verbatim}
running db nodes   = [ejabberd@first, ejabberd@second]
\end{verbatim}

  
\item Now run the following in the same ``\shell{erl}'' session:

\begin{verbatim}
mnesia:change_table_copy_type(schema, node(), disc_copies).
\end{verbatim}

  This will create local disc storage for DB.
  
  (alt) Change storage type of `\term{scheme}' table to ``RAM and disc
  copy'' on second node via web interface.

  
\item Now you can add replicas of various tables to this node with
  ``\verb|mnesia:add_table_copy|'' or
  ``\verb|mnesia:change_table_copy_type|'' as above (just replace
  ``\verb|schema|'' with another table name and ``\verb|disc_copies|''
  can be replaced with ``\verb|ram_copies|'' or
  ``\verb|disc_only_copies|'').
  
  What tables to replicate is very depend on your needs, you can get
  some hints from ``\verb|mnesia:info().|'' command, by looking at
  size of tables and default storage type for each table on 'first'.
  
  Replicating of table makes lookup in this table faster on this node,
  but writing will be slower.  And of course if machine with one of
  replicas is down, other replicas will be used.
  
  Also section 5.3 (Table Fragmentation) of
  \footahref{http://www.erlang.se/doc/doc-5.4/lib/mnesia-4.2/doc/html/index.html}
  {Mnesia Reference Manual} can be useful.
  
  (alt) Same as in previous item, but for other tables.

  
\item Run ``\verb|init:stop().|'' or just ``\verb|q().|'' to exit from
  erlang shell.  This probably can take some time if mnesia is not yet
  transfer and process all data it needed from \term{first}.

  
\item Now run ejabberd on \term{second} with almost the same config as
  on \term{first} (you probably don't need to duplicate ``\verb|acl|''
  and ``\verb|access|'' options --- they will be taken from
  \term{first}, and \verb|mod_muc| and \verb|mod_irc| should be
  enabled only on one machine in cluster).
\end{enumerate}

You can repeat these steps for other machines supposed to serve this
domain.


\appendix{}

\section{Built-in Modules}
\label{sec:modules}

\subsection{Common Options}
\label{sec:modcommonopts}

The following options are used by many modules, so they are described in
separate section.

\subsubsection{\option{iqdisc}}
\label{sec:modiqdiscoption}

Many modules define handlers for processing IQ queries of different namespaces
to this server or to user (e.\,g.\ to \jid{example.org} or to
\jid{user@example.org}).  This option defines processing discipline of
these queries.  Possible values are:
\begin{description}
\titem{no\_queue} All queries of namespace with this processing
  discipline processed immediately.  This also means that no other packets can
  be processed until finished this.  Hence this discipline is not recommended
  if processing of query can take relatively long time.
\titem{one\_queue} In this case created separate queue for processing
  of IQ queries of namespace with this discipline, and processing of this queue
  is done in parallel with processing of other packets. This discipline is most
  recommended.
\titem{parallel} In this case for all packets with this discipline
  spawned separate Erlang process, so all these packets processed in parallel.
  Although spawning of Erlang process have relatively low cost, this can broke
  server normal work, because Erlang emulator have limit on number of processes
  (32000 by default).
\end{description}

Example:
\begin{verbatim}
  {modules,
   [
    ...
    {mod_time, [{iqdisc, no_queue}]},
    ...
   ]}.
\end{verbatim}

\subsubsection{\option{host}}
\label{sec:modhostoption}

This option explicitly defines hostname for the module which acts as a service.

Example:
\begin{verbatim}
  {modules,
   [
    ...
    {mod_echo, [{host, "echo.example.org"}]},
    ...
   ]}.
\end{verbatim}

\subsubsection{\option{hosts}}
\label{sec:modhostsoption}

This option explicitly defines a list of hostnames for the module which acts as
a service.

Example:
\begin{verbatim}
  {modules,
   [
    ...
    {mod_echo, [{hosts, ["echo.example.org", "echo.example.com"]}]},
    ...
   ]}.
\end{verbatim}


\subsection{\modannounce{}}
\label{sec:modannounce}

This module adds support for broadcast announce messages and MOTD.
When the module is loaded, it handles messages sent to the following JID's
(suppose that main server has address \jid{example.org}):
\begin{description}
\titem{example.org/announce/all} Message is sent to all registered users at
\jid{example.org}.  If the user is online and connected to several resources,
only resource with the highest priority will receive the message.  If the
registered user is not connected, the message will be stored offline (if
oflline storage is available).
\titem{example.org/announce/online} Message is sent to all connected users at
\jid{example.org}.  If the user is online and connected to several resources,
all resources will receive the message.
\titem{example.org/announce/all-hosts/online} Message is sent to all connected
users at every virtual host.  If the user is online and connected to several
resources, all resources will receive the message.
\titem{example.org/announce/motd} Message is set as MOTD (Message of the Day)
and is sent to users at \jid{example.org} as they login.  In addition the
message is sent to all connected users (similar to \term{announce/online}
resource).
\titem{example.org/announce/motd/update} Message is set as MOTD (Message of the
Day) and is sent to users at \jid{example.org} as they login.  The message
is \emph{not sent} to all connected users.
\titem{example.org/announce/motd/delete} Any message sent to this JID
removes existing MOTD.
\end{description}

Options:
\begin{description}
\titem{access} Specifies who is allowed to send announce messages
and set MOTD (default value is \term{none}).
\end{description}

Example:
\begin{verbatim}
  % Only admins can send announcement messages:
  {access, announce, [{allow, admin}]}.

  {modules,
   [
    ...
    {mod_announce, [{access, announce}]},
    ...
   ]}.
\end{verbatim}


\subsection{\modconfigure{}}
\label{sec:modconfigure}

Options:
\begin{description}
\iqdiscitem{\ns{ejabberd:config}}
\end{description}


\subsection{\moddisco{}}
\label{sec:moddisco}

This module adds support for \jepref{0030} (Service Discovery).

Options:
\begin{description}
\iqdiscitem{\ns{http://jabber.org/protocol/disco\#items} and
  \ns{http://jabber.org/protocol/disco\#info}}
\titem{extra\_domains} List of domains that will be added to server
  items reply
\end{description}

Example:
\begin{verbatim}
  {modules,
   [
    ...
    {mod_disco, [{extra_domains, ["jit.example.com",
                                  "etc.example.com"]}]},
    ...
   ]}.
\end{verbatim}


\subsection{\modecho{}}
\label{sec:modecho}

This module acts as a service and simply returns to sender any \Jabber{}
packet.  Module may be useful for debugging.

Options:
\begin{description}
\hostitem{echo}
\end{description}


\subsection{\modirc{}}
\label{sec:modirc}

This module implements IRC transport.

Options:
\begin{description}
\hostitem{irc}
\titem{access} Specifies who is allowed to use IRC transport (default value is \term{all}).
\end{description}

Example:
\begin{verbatim}
  {modules,
   [
    ...
    {mod_irc, [{access, all}]},
    ...
   ]}.
\end{verbatim}


\subsection{\modlast{}}
\label{sec:modlast}

This module adds support for \jepref{0012} (Last Activity)

Options:
\begin{description}
\iqdiscitem{\ns{jabber:iq:last}}
\end{description}


\subsection{\modmuc{}}
\label{sec:modmuc}

This module implements \jepref{0045} (Multi-User Chat) service.

Options:
\begin{description}
\hostitem{conference}
\titem{access} Specifies who is allowed to use MUC service (default value is \term{all}).
\titem{access\_create} Specifies who is allowed to create new rooms at
  MUC service (default value is \term{all}).
\titem{access\_admin} Specifies who is allowed to administrate MUC service
(default value is \term{none}, which means that only creator may administer her room).
\end{description}

Example:
\begin{verbatim}
  % Define admin ACL
  {acl, admin, {user, "admin"}}

  % Define MUC admin access rule
  {access, muc_admin, [{allow, admin}]}

  {modules,
   [
    ...
    {mod_muc, [{access, all},
               {access_create, all},
               {access_admin, muc_admin}]},
    ...
   ]}.
\end{verbatim}


\subsection{\modoffline{}}
\label{sec:modoffline}

This module implements offline message storage.


\subsection{\modprivacy{}}
\label{sec:modprivacy}

This module implements Privacy Rules as defined in XMPP IM
(see \ahrefurl{http://www.jabber.org/ietf/}).

Options:
\begin{description}
\iqdiscitem{\ns{jabber:iq:privacy}}
\end{description}


\subsection{\modprivate{}}
\label{sec:modprivate}

This module adds support of \jepref{0049} (Private XML Storage).

Options:
\begin{description}
\iqdiscitem{\ns{jabber:iq:private}}
\end{description}


\subsection{\modpubsub{}}
\label{sec:modpubsub}

This module implements \jepref{0060} (Publish-Subscribe Service).

Options:
\begin{description}
\hostitem{pubsub}
\titem{served\_hosts} Specifies which hosts are served by the service.
If absent then only main \ejabberd{} host is served.
\end{description}

Example:
\begin{verbatim}
  {modules,
   [
    ...
    {mod_pubsub, [{served_hosts, ["example.com",
                                  "example.org"]}]}
    ...
   ]}.
\end{verbatim}


\subsection{\modregister{}}
\label{sec:modregister}

This module adds support for \jepref{0077} (In-Band Registration).

Options:
\begin{description}
\titem{access} Specifies rule to restrict registration.
If this rule returns ``deny'' on requested user name, then
registration is not allowed for it. (default value is \term{all}, which means
no restrictions).
\iqdiscitem{\ns{jabber:iq:register}}
\end{description}

Example:
\begin{verbatim}
  % Deny registration for users with too short name
  {acl, shortname, {user_glob, "?"}}.
  {acl, shortname, {user_glob, "??"}}.
  % Another variant: {acl, shortname, {user_regexp, "^..?$"}}.

  {access, register, [{deny, shortname},
                      {allow, all}]}.

  {modules,
   [
    ...
    {mod_register, [{access, register}]},
    ...
   ]}.
\end{verbatim}


\subsection{\modroster{}}
\label{sec:modroster}

This module implements roster management.

Options:
\begin{description}
\iqdiscitem{\ns{jabber:iq:roster}}
\end{description}


\subsection{\modservicelog{}}
\label{sec:modservicelog}

This module adds support for logging of user packets via any jabber service.
These packets encapsulated in <route/> element and sended to specified
services.

Options:
\begin{description}
  \titem{loggers} Specifies a list of services which will receive users
  packets.
\end{description}

Example:
\begin{verbatim}
  {modules,
   [
    ...
    {mod_service_log, [{loggers, ["bandersnatch.example.com"]}]},
    ...
   ]}.
\end{verbatim}


\subsection{\modsharedroster{}}
\label{sec:modsharedroster}

This module implements shared roster groups support.

You can edit shared roster groups via web-interface.  Each group has an unique
ID and the following parameters:
\begin{description}
\item[Name] The name of the group, which will be displayed in roster.
\item[Description] Textual description of this group, doesn't affect anything.
\item[Members] List of full JIDs of group members, entered one per line in
  web-interface.
\item[Displayed groups] List of IDs of groups which will be in rosters of this
  group members.
\end{description}

For example, to have a group of users which can see each other in roster,
create a group like on table~\ref{tab:srge1}.
\begin{table}[htbp]
  \centering
  \begin{tabular}{|l|l|}
    & Group `\texttt{users}'\\
    Name& Users\\
    Members&
    {\begin{tabular}{l}
        \jid{user1@example.org}\\
        \jid{user2@example.org}\\
        \jid{user3@example.org}
      \end{tabular}
    }\\
    Displayed groups& \texttt{users}
  \end{tabular}
  \caption{Shared group example N1}
  \label{tab:srge1}
\end{table}

To have 3 groups `\texttt{managers}', `\texttt{workgroup1}', and
`\texttt{workgroup2}', where group `\texttt{managers}' can see members of all
groups, and other two groups can see `\texttt{managers}' group and themselves,
create groups like on table~\ref{tab:srge2}.
\begin{table}[htbp]
  \centering
  \begin{tabular}{|l|l|l|l|}
    &
    Group `\texttt{managers}'&
    Group `\texttt{workgroup1}'&
    Group `\texttt{workgroup2}'\\
    Name& Managers& Workgroup1& Workgroup2\\
    Members&
    {\begin{tabular}{l}
        \jid{manager1@example.org}\\
        \jid{manager2@example.org}\\
        \jid{manager3@example.org}
      \end{tabular}
    }&
    {\begin{tabular}{l}
        \jid{user1@example.org}\\
        \jid{user2@example.org}\\
        \jid{user3@example.org}
      \end{tabular}
    }&
    {\begin{tabular}{l}
        \jid{user4@example.org}\\
        \jid{user5@example.org}\\
        \jid{user6@example.org}
      \end{tabular}
    }\\
    Displayed groups&
    {\begin{tabular}{l}
        \texttt{managers}\\
        \texttt{workgroup1}\\
        \texttt{workgroup2}
      \end{tabular}
    }&
    {\begin{tabular}{l}
        \texttt{managers}\\
        \texttt{workgroup1}
      \end{tabular}
    }&
    {\begin{tabular}{l}
        \texttt{managers}\\
        \texttt{workgroup2}
      \end{tabular}
    }
  \end{tabular}
  \caption{Shared group example N2}
  \label{tab:srge2}
\end{table}

\subsection{\modstats{}}
\label{sec:modstats}

This module adds support for \jepref{0039} (Statistics Gathering).

Options:
\begin{description}
\iqdiscitem{\ns{http://jabber.org/protocol/stats}}
\end{description}


\subsection{\modtime{}}
\label{sec:modtime}

This module answers UTC time on \ns{jabber:iq:time} queries.

Options:
\begin{description}
\iqdiscitem{\ns{jabber:iq:time}}
\end{description}


\subsection{\modvcard{}}
\label{sec:modvcard}

This module implements simple Jabber User Directory (based on user vCards)
and answers server vCard on \ns{vcard-temp} queries.

Options:
\begin{description}
\hostitem{vjud}
\iqdiscitem{\ns{vcard-temp}}
\titem{search} Specifies whether search is enabled (value is \term{true}, default) or
disabled (value is \term{false}) by the service. If \term{search} is set to \term{false},
option \term{host} is ignored and service does not appear in Jabber Discovery items.
\titem{matches} Limits the number of reported search results. If value is set to
\term{infinity} then all search results are reported.  Default value is \term{30}.
\titem{allow\_return\_all}  Specifies whether search with empty input fields can
return all known users.  Default is \term{false}.
\titem{search\_all\_hosts}  If set in \term{true} then search returns matched
items at all virtual hosts.  Otherwise only current host items are returned.
Default is \term{true}.
\end{description}

Example:
\begin{verbatim}
  {modules,
   [
    ...
    {mod_vcard, [{search, true},
                 {matches, 20},
                 {allow_return_all, true},
                 {search_all_hosts, false}]}
    ...
   ]}.
\end{verbatim}


\subsection{\modversion{}}
\label{sec:modversion}

This module answers \ejabberd{} version on \ns{jabber:iq:version} queries.

Options:
\begin{description}
\iqdiscitem{\ns{jabber:iq:version}}
\end{description}


\section{I18n/L10n}
\label{sec:i18nl10n}

All built-in modules support \texttt{xml:lang} attribute inside IQ queries.
E.\,g.\ on figure~\ref{fig:discorus} showed the reply on the following query:
\begin{verbatim}
  <iq id='5'
      to='example.org'
      type='get'
      xml:lang='ru'>
    <query xmlns='http://jabber.org/protocol/disco#items'/>
  </iq>
\end{verbatim}

\begin{figure}[htbp]
  \centering
  \insimg{discorus.png}
  \caption{Discovery result when \texttt{xml:lang='ru'}}
  \label{fig:discorus}
\end{figure}

Also web-interface supports \verb|Accept-Language| HTTP header (see
figure~\ref{fig:webadmmainru}, compare it with figure~\ref{fig:webadmmain})

\begin{figure}[htbp]
  \centering
  \insimg{webadmmainru.png}
  \caption{Web-administration top page with HTTP header
    ``\verb|Accept-Language: ru|''}
  \label{fig:webadmmainru}
\end{figure}


\end{document}
