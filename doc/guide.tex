\documentclass[a4paper,10pt]{book}

%% Packages
\usepackage{float}
\usepackage{graphics}
\usepackage{hevea}
\usepackage[pdftex,colorlinks,unicode,urlcolor=blue,linkcolor=blue,
        pdftitle=Ejabberd\ Installation\ and\ Operation\ Guide,pdfauthor=Process-one,pdfsubject=ejabberd,pdfkeywords=ejabberd,
        pdfpagelabels=false]{hyperref}
\usepackage{makeidx}
%\usepackage{showidx} % Only for verifying the index entries.
\usepackage{verbatim}
\usepackage{geometry}
\usepackage{fancyhdr}

\pagestyle{fancy}                         %Forces the page to use the fancy template
\renewcommand{\chaptermark}[1]{\markboth{\textbf{\thechapter}.\ \emph{#1}}{}}
\renewcommand{\sectionmark}[1]{\markright{\thesection\ \boldmath\textbf{#1}\unboldmath}}
\fancyhf{}
\fancyhead[LE,RO]{\textbf{\thepage}}      %Displays the page number in bold in the header,
                                           % to the left on even pages and to the right on odd pages.
\fancyhead[RE]{\nouppercase{\leftmark}}    %Displays the upper-level (chapter) information---
                                           % as determined above---in non-upper case in the header, to the right on even pages.
\fancyhead[LO]{\rightmark}                 %Displays the lower-level (section) information---as
                                           % determined above---in the header, to the left on odd pages.
\renewcommand{\headrulewidth}{0.5pt}       %Underlines the header. (Set to 0pt if not required).
\renewcommand{\footrulewidth}{0.5pt}       %Underlines the footer. (Set to 0pt if not required).

%% Index
\makeindex
% Remove the index anchors from the HTML version to save size and bandwith.
\newcommand{\ind}[1]{\begin{latexonly}\index{#1}\end{latexonly}}

%% Images
\newcommand{\logoscale}{0.7}
\newcommand{\imgscale}{0.58}
\newcommand{\insimg}[1]{\insscaleimg{\imgscale}{#1}}
\newcommand{\insscaleimg}[2]{
  \imgsrc{#2}{}
  \begin{latexonly}
    \scalebox{#1}{\includegraphics{#2}}
  \end{latexonly}
}

%% Various
\newcommand{\bracehack}{\def\{{\char"7B}\def\}{\char"7D}}
\newcommand{\titem}[1]{\item[\bracehack\texttt{#1}]}
\newcommand{\ns}[1]{\texttt{#1}}
\newcommand{\jid}[1]{\texttt{#1}}
\newcommand{\option}[1]{\texttt{#1}}
\newcommand{\poption}[1]{{\bracehack\texttt{#1}}}
\newcommand{\node}[1]{\texttt{#1}}
\newcommand{\term}[1]{\texttt{#1}}
\newcommand{\shell}[1]{\texttt{#1}}
\newcommand{\ejabberd}{\texttt{ejabberd}}
\newcommand{\Jabber}{Jabber}

%% Modules
\newcommand{\module}[1]{\texttt{#1}}
\newcommand{\modadhoc}{\module{mod\_adhoc}}
\newcommand{\modannounce}{\module{mod\_announce}}
\newcommand{\modconfigure}{\module{mod\_configure}}
\newcommand{\moddisco}{\module{mod\_disco}}
\newcommand{\modecho}{\module{mod\_echo}}
\newcommand{\modirc}{\module{mod\_irc}}
\newcommand{\modlast}{\module{mod\_last}}
\newcommand{\modlastodbc}{\module{mod\_last\_odbc}}
\newcommand{\modmuc}{\module{mod\_muc}}
\newcommand{\modmuclog}{\module{mod\_muc\_log}}
\newcommand{\modoffline}{\module{mod\_offline}}
\newcommand{\modofflineodbc}{\module{mod\_offline\_odbc}}
\newcommand{\modprivacy}{\module{mod\_privacy}}
\newcommand{\modprivate}{\module{mod\_private}}
\newcommand{\modprivateodbc}{\module{mod\_private\_odbc}}
\newcommand{\modproxy}{\module{mod\_proxy65}}
\newcommand{\modpubsub}{\module{mod\_pubsub}}
\newcommand{\modregister}{\module{mod\_register}}
\newcommand{\modroster}{\module{mod\_roster}}
\newcommand{\modrosterodbc}{\module{mod\_roster\_odbc}}
\newcommand{\modservicelog}{\module{mod\_service\_log}}
\newcommand{\modsharedroster}{\module{mod\_shared\_roster}}
\newcommand{\modstats}{\module{mod\_stats}}
\newcommand{\modtime}{\module{mod\_time}}
\newcommand{\modvcard}{\module{mod\_vcard}}
\newcommand{\modvcardldap}{\module{mod\_vcard\_ldap}}
\newcommand{\modvcardodbc}{\module{mod\_vcard\_odbc}}
\newcommand{\modversion}{\module{mod\_version}}

%% Common options
\newcommand{\iqdiscitem}[1]{\titem{iqdisc} \ind{options!iqdisc}This specifies 
the processing discipline for #1 IQ queries (see section~\ref{modiqdiscoption}).}
\newcommand{\hostitem}[1]{
  \titem{hosts} \ind{options!hosts} This option defines the hostnames of the
  service (see section~\ref{modhostsoption}). If neither \texttt{hosts} nor
  the old \texttt{host} is present, the prefix `\jid{#1.}' is added to all
  \ejabberd{} hostnames.
}

%% Title page
% ejabberd version (automatically generated).
\newcommand{\version}{2.0.0b1}

\newlength{\larg}
\setlength{\larg}{14.5cm}
\title{
{\rule{\larg}{1mm}}\vspace{7mm}
\begin{tabular}{r}
    {\huge {\bf ejabberd \version\ }} \\
    \\
    {\huge Installation and Operation Guide}
\end{tabular}\\
\vspace{2mm}
{\rule{\larg}{1mm}}
\vspace{2mm} \\
\begin{tabular}{r}
    {\large  \bf \today}
\end{tabular}\\
\vspace{5.5cm}
}
\author{\begin{tabular}{p{13.7cm}}
ejabberd Development Team
\end{tabular}}
\date{}


%% Options
\newcommand{\marking}[1]{#1} % Marking disabled
\newcommand{\quoting}[2][yozhik]{} % Quotes disabled
%\newcommand{\new}{\marginpar{\textsc{new}}} % Highlight new features
%\newcommand{\improved}{\marginpar{\textsc{improved}}} % Highlight improved features

%% To by-pass errors in the HTML version.
\newstyle{SPAN}{width:20\%; float:right; text-align:left; margin-left:auto;}

%% Footnotes
\begin{latexonly}
\global\parskip=9pt plus 3pt minus 1pt
\global\parindent=0pt
\gdef\ahrefurl#1{\href{#1}{\texttt{#1}}}
\gdef\footahref#1#2{#2\footnote{\href{#1}{\texttt{#1}}}}
\end{latexonly}
\newcommand{\txepref}[2]{\footahref{http://www.xmpp.org/extensions/xep-#1.html}{#2}}
\newcommand{\xepref}[1]{\txepref{#1}{XEP-#1}}

\begin{document}

\label{titlepage}
\begin{titlepage}
  \maketitle{}

%% Commenting. Breaking clean layout for now:
%%  \begin{center}
%%  {\insscaleimg{\logoscale}{logo.png}
%%    \par
%%  }
%%  \end{center}

%%  \begin{quotation}\textit{I can thoroughly recommend ejabberd for ease of setup ---
%%  Kevin Smith, Current maintainer of the Psi project}\end{quotation}

\end{titlepage}

% Set the page counter to 2 so that the titlepage and the second page do not
% have the same page number. This fixes the PDFLaTeX warning "destination with
% the same identifier".
\begin{latexonly}
\setcounter{page}{2}
\end{latexonly}

\tableofcontents{}

% Input introduction.tex
\section{Introduction}
\label{intro}

\quoting{I just tried out ejabberd and was impressed both by ejabberd itself and the language it is written in, Erlang. ---
Joeri} 

%ejabberd is a free and open source instant messaging server written in Erlang. ejabberd is cross-platform, distributed, fault-tolerant, and based on open standards to achieve real-time communication (Jabber/XMPP).

\ejabberd{} is a \marking{free and open source} instant messaging server written in \footahref{http://www.erlang.org/}{Erlang}.

\ejabberd{} is \marking{cross-platform}, distributed, fault-tolerant, and based on open standards to achieve real-time communication.

\ejabberd{} is designed to be a \marking{rock-solid and feature rich} XMPP server.

\ejabberd{} is suitable for small deployments, whether they need to be \marking{scalable} or not, as well as extremely big deployments.

%\subsection{Layout with example deployment (title needs a better name)}
%\label{layout}

%In this section there will be a graphical overview like these:\\
%\verb|http://www.tipic.com/var/timp/timp_dep.gif| \\
%\verb|http://www.jabber.com/images/jabber_Com_Platform.jpg| \\
%\verb|http://www.antepo.com/files/OPN45systemdatasheet.pdf| \\

%A page full with names of Jabber client that are known to work with ejabberd. \begin{tiny}tiny font\end{tiny}

%\subsection{Try It Today}
%\label{trytoday}

%(Not sure if I will include/finish this section for the next version.)

%\begin{itemize}
%\item Erlang REPOS
%\item Packages in distributions
%\item Windows binary
%\item source tar.gz
%\item Migration from Jabberd14 (and so also Jabberd2 because you can migrate from version 2 back to 14) and Jabber Inc. XCP possible.
%\end{itemize}

\newpage
\subsection{Key Features}
\label{keyfeatures}
\ind{features!key features}

\quoting{Erlang seems to be tailor-made for writing stable, robust servers. ---
Peter Saint-Andr\'e, Executive Director of the Jabber Software Foundation}

\ejabberd{} is:
\begin{itemize}
\item \marking{Cross-platform:} \ejabberd{} runs under Microsoft Windows and Unix derived systems such as Linux, FreeBSD and NetBSD.

\item \marking{Distributed:} You can run \ejabberd{} on a cluster of machines and all of them will serve the same \Jabber{} domain(s). When you need more capacity you can simply add a new cheap node to your cluster. Accordingly, you do not need to buy an expensive high-end machine to support tens of thousands concurrent users.

\item \marking{Fault-tolerant:} You can deploy an \ejabberd{} cluster so that all the information required for a properly working service will be replicated permanently on all nodes. This means that if one of the nodes crashes, the others will continue working without disruption. In addition, nodes also can be added or replaced `on the fly'.

\item \marking{Administrator Friendly:} \ejabberd{} is built on top of the Open Source Erlang. As a result you do not need to install an external database, an external web server, amongst others because everything is already included, and ready to run out of the box. Other administrator benefits include:
\begin{itemize}
\item Comprehensive documentation.
\item Straightforward installers for Linux, Mac OS X, and Windows.\improved{}
\item Web interface for administration tasks.
\item Shared Roster Groups.
\item Command line administration tool.\improved{}
\item Can integrate with existing authentication mechanisms.
\item Capability to send announce messages.
\end{itemize}

\item \marking{Internationalized:} \ejabberd{} leads in internationalization. Hence it is very well suited in a globalized world. Related features are:
\begin{itemize}
\item Translated in 12 languages.\improved{}
\item Support for \footahref{http://www.ietf.org/rfc/rfc3490.txt}{IDNA}.
\end{itemize}

\item \marking{Open Standards:} \ejabberd{} is the first Open Source Jabber server claiming to fully comply to the XMPP standard.
\begin{itemize}
\item Fully XMPP compliant.
\item XML-based protocol.
\item \footahref{http://ejabberd.jabber.ru/protocols}{Many protocols supported}.
\end{itemize}

\end{itemize}

\newpage

\subsection{Additional Features}
\label{addfeatures}
\ind{features!additional features}

\quoting{ejabberd is making inroads to solving the "buggy incomplete server" problem ---
Justin Karneges, Founder of the Psi and the Delta projects}

Moreover, \ejabberd{} comes with a wide range of other state-of-the-art features:
\begin{itemize}
\item Modular
\begin{itemize}
\item Load only the modules you want.
\item Extend \ejabberd{} with your own custom modules.
\end{itemize}
\item Security
\begin{itemize}
\item SASL and STARTTLS for c2s and s2s connections.
\item STARTTLS and Dialback s2s connections.
\item Web interface accessible via HTTPS secure access.
\end{itemize}
\item Databases
\begin{itemize}
\item Native MySQL support.
\item Native PostgreSQL support.
\item Mnesia.
\item ODBC data storage support.
\item Microsoft SQL Server support.\new{}
\end{itemize}
\item Authentication
\begin{itemize}
\item LDAP and ODBC.\improved{}
\item External Authentication script.
\item Internal Authentication.
\end{itemize}
\item Others
\begin{itemize}
\item Compressing XML streams with Stream Compression (\xepref{0138}).
\item Interface with networks such as AIM, ICQ and MSN.
\item Statistics via Statistics Gathering (\xepref{0039}).
\item IPv6 support both for c2s and s2s connections.
\item \txepref{0045}{Multi-User Chat} module with logging.\improved{}
\item Users Directory based on users vCards.
\item \txepref{0060}{Publish-Subscribe} component.
\item Support for virtual hosting.
\item \txepref{0025}{HTTP Polling} service.
\item IRC transport.
\end{itemize}
\end{itemize}

\chapter{Installing ejabberd}
\section{Installing ejabberd with Graphical Installer}

The easiest approach to install an ejabberd Instant Messaging server
is to use the graphical installer. The installer is available in
ejabberd Process-one 
\footahref{http://www.process-one.net/en/ejabberd/downloads/}{downloads page}.

The installer will deploy and configure a full featured ejabberd
server and does not require any extra dependancies.

\section{Installing ejabberd with Operating System specific packages}

Some Operating Systems provide a specific ejabberd package adapted to 
your system architecture and libraries, which also checks dependencies 
and performs basic configuration tasks like creating the initial
administrator account. Some examples are Debian and Gentoo. Consult the
resources provided by your Operating System for more information.

\section{Installing ejabberd with CEAN}

\footahref{http://cean.process-one.net/}{CEAN}
(Comprehensive Erlang Archive Network) is a repository that hosts binary
packages from many Erlang programs, including ejabberd and all its dependencies.
The binaries are available for many different system architectures, so this is an
alternative to the binary installer and Operating System's ejabberd packages.


\section{Installation from Source}
\label{installsource}
\ind{installation}

\subsection{Installation Requirements}
\label{installreq}
\ind{installation!requirements}

\subsubsection{`Unix-like' operating systems}
\label{installrequnix}

To compile \ejabberd{} on a `Unix-like' operating system, you need:
\begin{itemize}
\item GNU Make
\item GCC
\item libexpat 1.95 or higher
\item Erlang/OTP R9C-2 or higher
\item OpenSSL 0.9.6 or higher (optional)
\item Zlib 1.2.3 or higher (optional)
\item GNU Iconv 1.8 or higher (optional, not needed on systems with GNU libc)
\end{itemize}

\subsubsection{Windows}
\label{installreqwin}

To compile \ejabberd{} on a Windows flavour, you need:
\begin{itemize}
\item MS Visual C++ 6.0 Compiler
\item \footahref{http://erlang.org/download.html}{Erlang/OTP R9C-2 or higher}
\item \footahref{http://sourceforge.net/project/showfiles.php?group\_id=10127\&package\_id=11277}{Expat 1.95.7 or higher}
\item
\footahref{http://www.gnu.org/software/libiconv/}{GNU Iconv 1.9.1}
(optional)
\item \footahref{http://www.slproweb.com/products/Win32OpenSSL.html}{Shining Light OpenSSL}
(to enable SSL connections)
\item \footahref{http://www.zlib.net/}{Zlib 1.2.3 or higher}
\end{itemize}

\subsection{Obtaining \ejabberd{}}
\label{obtaining}

\ind{download}
Released versions of \ejabberd{} can be obtained from \\
\ahrefurl{http://www.process-one.net/en/projects/ejabberd/download.html}.

\ind{Subversion repository}
The latest development version can be retrieved from the Subversion repository\@.
\begin{verbatim}
  svn co http://svn.process-one.net/ejabberd/trunk ejabberd
\end{verbatim}

\subsection{Compilation}
\label{compile}
\ind{installation!compilation}

\subsubsection{`Unix-like' operating systems}
\label{compileunix}

Compile \ejabberd{} on a `Unix-like' operating system by executing:

\begin{verbatim}
  ./configure
  make
  su
  make install
\end{verbatim}

These commands will:
\begin{itemize}
\item install \ejabberd{} into the directory \verb|/var/lib/ejabberd|,
\item install the configuration file into \verb|/etc/ejabberd|,
\item create a directory called \verb|/var/log/ejabberd| to store log files.
\end{itemize}

\subsubsection{Compilation options}

If you want to use an external database, you need to execute the configure
script with the option(s) \term{--enable-odbc} or \term{--enable-odbc
--enable-mssql}. See section~\ref{database} for more information.

\subsubsection{Windows}
\label{compilewin}

\begin{itemize}
\item Install Erlang emulator (for example, into \verb|C:\Program Files\erl5.3|).
\item Install Expat library into \verb|C:\Program Files\Expat-1.95.7|
  directory.

  Copy file \verb|C:\Program Files\Expat-1.95.7\Libs\libexpat.dll|
  to your Windows system directory (for example, \verb|C:\WINNT| or
  \verb|C:\WINNT\System32|)
\item Build and install the Iconv library into the directory
  \verb|C:\Program Files\iconv-1.9.1|.

  Copy file \verb|C:\Program Files\iconv-1.9.1\bin\iconv.dll| to your
  Windows system directory (more installation instructions can be found in the
  file README.woe32 in the iconv distribution).

  Note: instead of copying libexpat.dll and iconv.dll to the Windows
  directory, you can add the directories
  \verb|C:\Program Files\Expat-1.95.7\Libs| and
  \verb|C:\Program Files\iconv-1.9.1\bin| to the \verb|PATH| environment
  variable.
\item While in the directory \verb|ejabberd\src| run:
\begin{verbatim}
configure.bat
nmake -f Makefile.win32
\end{verbatim}
\item Edit the file \verb|ejabberd\src\ejabberd.cfg| and run
\begin{verbatim}
werl -s ejabberd -name ejabberd
\end{verbatim}
\end{itemize}

%TODO: how to compile database support on windows?

\subsection{Starting}
\label{start}
\ind{starting}
%TODO: update when the ejabberdctl script is made more userfriendly

Execute the following command to start \ejabberd{}:
\begin{verbatim}
  erl -pa /var/lib/ejabberd/ebin -name ejabberd -s ejabberd
\end{verbatim}
or
\begin{verbatim}
  erl -pa /var/lib/ejabberd/ebin -sname ejabberd -s ejabberd
\end{verbatim}
In the latter case the Erlang node will be identified using only the first part
of the host name, i.\,e. other Erlang nodes outside this domain cannot contact
this node.

Note that when using the above command, \ejabberd{} will search for the
configuration file in the current directory and will use the current directory
for storing its user database and for logging.

To specify the path to the configuration file, the log files and the Mnesia
database directory, you may use the following command:
\begin{verbatim}
  erl -pa /var/lib/ejabberd/ebin \
      -sname ejabberd \
      -s ejabberd \
      -ejabberd config \"/etc/ejabberd/ejabberd.cfg\" \
                log_path \"/var/log/ejabberd/ejabberd.log\" \
      -sasl sasl_error_logger \{file,\"/var/log/ejabberd/sasl.log\"\} \
      -mnesia dir \"/var/lib/ejabberd/spool\"
\end{verbatim}

You can find other useful options in the Erlang manual page
(\shell{erl -man erl}).

To use more than 1024 connections, you should set the environment variable
\verb|ERL_MAX_PORTS|:
\begin{verbatim}
  export ERL_MAX_PORTS=32000
\end{verbatim}
Note that with this value, \ejabberd{} will use more memory (approximately 6\,MB
more).

To reduce memory usage, you may set the environment variable
\verb|ERL_FULLSWEEP_AFTER|:
\begin{verbatim}
  export ERL_FULLSWEEP_AFTER=0
\end{verbatim}
But in this case \ejabberd{} can start to work slower.

\section{Creating an Initial Administrator}
\label{initialadmin}

Before the web interface can be entered to perform administration tasks, an
account with administrator rights is needed on your \ejabberd{} deployment.

Instructions to create an initial administrator account:
\begin{enumerate}
\item Register an account on your \ejabberd{} deployment. An account can be
  created in two ways:
  \begin{enumerate}
  \item Using the tool \term{ejabberdctl}\ind{ejabberdctl} (see
    section~\ref{ejabberdctl}):
    \begin{verbatim}
% ejabberdctl node@host register admin example.org password
\end{verbatim} 
  \item Using In-Band Registration (see section~\ref{modregister}): you can
    use a \Jabber{} client to register an account.
  \end{enumerate}
\item Edit the configuration file to promote the account created in the previous
  step to an account with administrator rights. Note that if you want to add
  more administrators, a seperate acl entry is needed for each administrator.
  \begin{verbatim}
  {acl, admins, {user, "admin", "example.org"}}.
  {access, configure, [{allow, admins}]}.
\end{verbatim} 
\item Restart \ejabberd{} to load the new configuration.
\item Open the web interface (\verb|http://server:port/admin/|) in your
  favourite browser. Make sure to enter the \emph{full} JID as username (in this
  example: \jid{admin@example.org}. The reason that you also need to enter the
  suffix, is because \ejabberd{}'s virtual hosting support.
\end{enumerate}


\chapter{Configuring ejabberd}
\section{Basic Configuration}
\label{basicconfig}
\ind{configuration file}

The configuration file will be loaded the first time you start \ejabberd{}. The
content from this file will be parsed and stored in a database. Subsequently the
configuration will be loaded from the database and any commands in the
configuration file are appended to the entries in the database. The
configuration file contains a sequence of Erlang terms. Lines beginning with a
\term{`\%'} sign are ignored. Each term is a tuple of which the first element is
the name of an option, and any further elements are that option's values. If the
configuration file do not contain for instance the `hosts' option, the old
host name(s) stored in the database will be used.


You can override the old values stored in the database by adding next lines to
the configuration file:
\begin{verbatim}
  override_global.
  override_local.
  override_acls.
\end{verbatim}
With these lines the old global options (shared between all ejabberd nodes in a
cluster), local options (which are specific for this particular ejabberd node)
and ACLs will be removed before new ones are added.

\subsection{Host Names}
\label{hostnames}
\ind{options!hosts}\ind{host names}

The option \option{hosts} defines a list containing one or more domains that
\ejabberd{} will serve.

Examples:
\begin{itemize}
\item Serving one domain:
  \begin{verbatim}
  {hosts, ["example.org"]}.
\end{verbatim}
\item Serving one domain, and backwards compatible with older \ejabberd{}
  versions:
  \begin{verbatim}
  {host, "example.org"}.
\end{verbatim}
\item Serving two domains:
\begin{verbatim}
  {hosts, ["example.net", "example.com"]}.
\end{verbatim}
\end{itemize}

\subsection{Virtual Hosting}
\label{virtualhost}
\ind{virtual hosting}\ind{virtual hosts}\ind{virtual domains}

Options can be defined separately for every virtual host using the
\term{host\_config} option.\ind{options!host\_config} It has the following
syntax:
\begin{verbatim}
  {host_config, <hostname>, [<option>, <option>, ...]}.
\end{verbatim}

Examples:
\begin{itemize}
\item Domain \jid{example.net} is using the internal authentication method while
  domain \jid{example.com} is using the \ind{LDAP}LDAP server running on the
  domain \jid{localhost} to perform authentication:
\begin{verbatim}
{host_config, "example.net", [{auth_method, internal}]}.

{host_config, "example.com", [{auth_method, ldap},
                              {ldap_servers, ["localhost"]},
                              {ldap_uids, [{"uid"}]},
                              {ldap_rootdn, "dc=localdomain"},
                              {ldap_rootdn, "dc=example,dc=com"},
                              {ldap_password, ""}]}.
\end{verbatim}
\item Domain \jid{example.net} is using \ind{ODBC}ODBC to perform authentication
  while domain \jid{example.com} is using the LDAP servers running on the domains
  \jid{localhost} and \jid{otherhost}:
\begin{verbatim}
{host_config, "example.net", [{auth_method, odbc},
                              {odbc_server, "DSN=ejabberd;UID=ejabberd;PWD=ejabberd"}]}.

{host_config, "example.com", [{auth_method, ldap},
                              {ldap_servers, ["localhost", "otherhost"]},
                              {ldap_uids, [{"uid"}]},
                              {ldap_rootdn, "dc=localdomain"},
                              {ldap_rootdn, "dc=example,dc=com"},
                              {ldap_password, ""}]}.
\end{verbatim}
\end{itemize}

\subsection{Listened Sockets}
\label{listened}
\ind{options!listen}

The option \option{listen} defines for which addresses and ports \ejabberd{}
will listen and what services will be run on them. Each element of the list is a
tuple with the following elements:
\begin{itemize}
\item Port number.
\item Module that serves this port.
\item Options to this module.
\end{itemize}

\ind{modules!ejabberd\_c2s}\ind{modules!ejabberd\_s2s\_in}\ind{modules!ejabberd\_service}\ind{modules!ejabberd\_http}\ind{protocols!XEP-0114: Jabber Component Protocol}
Currently next modules are implemented:
\begin{table}[H]
  \centering
  \def\arraystretch{1.4}
  \begin{tabular}{|l|l|p{87mm}|}
  \hline \texttt{ejabberd\_c2s}& Description& Handles c2s connections.\\
    \cline{2-3} & Options& \texttt{access}, \texttt{certfile}, \texttt{inet6},
    \texttt{ip}, \texttt{max\_stanza\_size}, \texttt{shaper}, \texttt{ssl},
    \texttt{tls}, \texttt{starttls}, \texttt{starttls\_required},
    \texttt{zlib}\\
  \hline \texttt{ejabberd\_s2s\_in}& Description& Handles incoming s2s
  connections.\\
    \cline{2-3} & Options& \texttt{inet6}, \texttt{ip},
    \texttt{max\_stanza\_size}\\
  \hline \texttt{ejabberd\_service}& Description& Interacts with external
  components (*).\\
    \cline{2-3} & Options& \texttt{access}, \texttt{hosts}, \texttt{inet6},
    \texttt{ip}, \texttt{shaper}\\
  \hline \texttt{ejabberd\_http}& Description& Handles incoming HTTP
  connections.\\
    \cline{2-3} & Options& \texttt{certfile}, \texttt{http\_poll},
    \texttt{inet6}, \texttt{ip}, \texttt{tls}, \texttt{web\_admin}\\
  \hline
  \end{tabular}
\end{table}

(*) The mechanism for \footahref{http://ejabberd.jabber.ru/tutorials-transports}{external components} is defined in Jabber Component Protocol (\xepref{0114}).

The following options are available:
\begin{description}
  \titem{\{access, <access rule>\}} \ind{options!access}This option defines
    access to the port. The default value is \term{all}.
  \titem{\{certfile, Path\}} Path to a file containing the SSL certificate.
  \titem{\{hosts, [Hostnames], [HostOptions]\}} \ind{options!hosts}This option
    defines one or more hostnames of connected services and enables you to
    specify additional options including \poption{\{password, Secret\}}.
  \titem{http\_poll} \ind{options!http\_poll}\ind{protocols!XEP-0025: HTTP Polling}\ind{JWChat}\ind{web-based Jabber client}
    This option enables HTTP Polling (\xepref{0025}) support. HTTP Polling
    enables access via HTTP requests to \ejabberd{} from behind firewalls which
    do not allow outgoing sockets on port 5222.

    If HTTP Polling is enabled, it will be available at
    \verb|http://server:port/http-poll/|. Be aware that support for HTTP Polling
    is also needed in the \Jabber{} client. Remark also that HTTP Polling can be
    interesting to host a web-based \Jabber{} client such as
    \footahref{http://jwchat.sourceforge.net/}{JWChat} (there is a tutorial to
    \footahref{http://ejabberd.jabber.ru/jwchat}{install JWChat} with
    instructions for \ejabberd{}).
  \titem{inet6} \ind{options!inet6}\ind{IPv6}Set up the socket for IPv6.
  \titem{\{ip, IPAddress\}} \ind{options!ip}This option specifies which network
    interface to listen for. For example \verb|{ip, {192, 168, 1, 1}}|.
    \titem{\{max\_stanza\_size, Size\}}
    \ind{options!max\_stanza\_size}This option specifies an
    approximate maximum size in bytes of XML stanzas.  Approximate,
    because it is calculated with the precision of one block of readed
    data. For example \verb|{max_stanza_size, 65536}|.  The default
    value is \term{infinity}. Recommanded values are 65536 for c2s
    connections and 131072 for s2s connections. s2s max stanza size
    must always much higher than c2s limit. Change this value with
    extreme care as it can cause unwanted disconnect if set too low.
  \titem{\{shaper, <access rule>\}} \ind{options!shaper}This option defines a
    shaper for the port (see section~\ref{shapers}). The default value
    is \term{none}.
  \titem{ssl} \ind{options!ssl}\ind{SSL}This option specifies that traffic on
    the port will be encrypted using SSL. You should also set the
    \option{certfile} option. It is recommended to use the \term{tls} option
    instead.
  \titem{starttls} \ind{options!starttls}\ind{STARTTLS}This option
    specifies that STARTTLS encryption is available on connections to the port.
    You should also set the \option{certfile} option.
  \titem{starttls\_required} \ind{options!starttls\_required}This option
    specifies that STARTTLS encryption is required on connections to the port.
    No unencrypted connections will be allowed. You should also set the
    \option{certfile} option.
  \titem{tls} \ind{options!tls}\ind{TLS}This option specifies that traffic on
    the port will be encrypted using SSL immediately after connecting. You
    should also set the \option{certfile} option.
  \titem{zlib} \ind{options!zlib}\ind{protocols!XEP-0138: Stream Compression}\ind{Zlib}This
    option specifies that Zlib stream compression (as defined in \xepref{0138})
    is available on connections to the port. Client connections cannot use
    stream compression and stream encryption simultaneously. Hence, if you
    specify both \option{tls} (or \option{ssl}) and \option{zlib}, the latter
    option will not affect connections (there will be no stream compression).
  \titem{web\_admin} \ind{options!web\_admin}\ind{web interface}This option
    enables the web interface for \ejabberd{} administration which is available
    at \verb|http://server:port/admin/|. Login and password are the username and
    password of one of the registered users who are granted access by the
    `configure' access rule.
    \titem{component\_check\_from} \ind{options!service\_check\_from}
    This option can be used with \term{ejabberd\_service} only. It is
    used to disable control on the from field on packets send by an
    external components. The option can be either \term{true} or
    \term{false}. The default value is \term{true} which conforms to \xepref{0114}.
\end{description}

In addition, the following options are available for s2s connections:
\begin{description}
  \titem{\{s2s\_use\_starttls, true|false\}}
  \ind{options!s2s\_use\_starttls}\ind{STARTTLS}This option defines whether to
  use STARTTLS for s2s connections.
  \titem{\{s2s\_certfile, Path\}} \ind{options!s2s\_certificate}Path to a
  file containing a SSL certificate.
  \titem{\{domain\_certfile, Domain, Path\}} \ind{options!domain\_certfile}Path
  to the file containing the SSL certificate for the specified domain.
\end{description}

For instance, the following configuration defines that:
\begin{itemize}
\item c2s connections are listened for on port 5222 and 5223 (SSL) and denied
  for the user called `\term{bad}'.
\item s2s connections are listened for on port 5269 with STARTTLS for secured
  traffic enabled.
\item Port 5280 is serving the web interface and the HTTP Polling service. Note
  that it is also possible to serve them on different ports. The second
  example in section~\ref{webinterface} shows how exactly this can be done.
\item All users except for the administrators have a traffic of limit 
  1,000\,Bytes/second
\item \ind{transports!AIM}The
  \footahref{http://ejabberd.jabber.ru/pyaimt}{AIM transport}
  \jid{aim.example.org} is connected to port 5233 with password
  `\term{aimsecret}'.
\item \ind{transports!ICQ}The ICQ transport JIT (\jid{icq.example.org} and
  \jid{sms.example.org}) is connected to port 5234 with password
  `\term{jitsecret}'.
\item \ind{transports!MSN}The
  \footahref{http://ejabberd.jabber.ru/pymsnt}{MSN transport}
  \jid{msn.example.org} is connected to port 5235 with password
  `\term{msnsecret}'.
\item \ind{transports!Yahoo}The
  \footahref{http://ejabberd.jabber.ru/yahoo-transport-2}{Yahoo! transport}
  \jid{yahoo.example.org} is connected to port 5236 with password
  `\term{yahoosecret}'.
\item \ind{transports!Gadu-Gadu}The \footahref{http://ejabberd.jabber.ru/jabber-gg-transport}{Gadu-Gadu transport} \jid{gg.example.org} is
  connected to port 5237 with password `\term{ggsecret}'.
\item \ind{transports!email notifier}The
  \footahref{http://ejabberd.jabber.ru/jmc}{Jabber Mail Component}
  \jid{jmc.example.org} is connected to port 5238 with password
  `\term{jmcsecret}'.
\item The service custom has enabled the special option to avoiding checking the \term{from} attribute in the packets send by this component. The component can send packets in behalf of any users from the server, or even on behalf of any server.
\end{itemize}
\begin{verbatim}
  {acl, blocked, {user, "bad"}}.
  {access, c2s, [{deny, blocked},
                 {allow, all}]}.
  {shaper, normal, {maxrate, 1000}}.
  {access, c2s_shaper, [{none, admin},
                        {normal, all}]}.
  {listen,
   [{5222, ejabberd_c2s,     [{access, c2s}, {shaper, c2s_shaper}]},
    {5223, ejabberd_c2s,     [{access, c2s},
                              ssl, {certfile, "/path/to/ssl.pem"}]},
    {5269, ejabberd_s2s_in,  []},
    {5280, ejabberd_http,    [http_poll, web_admin]},
    {5233, ejabberd_service, [{host, "aim.example.org",
                               [{password, "aimsecret"}]}]},
    {5234, ejabberd_service, [{hosts, ["icq.example.org", "sms.example.org"],
                               [{password, "jitsecret"}]}]},
    {5235, ejabberd_service, [{host, "msn.example.org",
                               [{password, "msnsecret"}]}]},
    {5236, ejabberd_service, [{host, "yahoo.example.org",
                               [{password, "yahoosecret"}]}]},
    {5237, ejabberd_service, [{host, "gg.example.org",
                               [{password, "ggsecret"}]}]},
    {5238, ejabberd_service, [{host, "jmc.example.org",
                               [{password, "jmcsecret"}]}]},
    {5239, ejabberd_service, [{host, "custom.example.org",
                               [{password, "customsecret"}]},
                              {service_check_from, false}]}
   ]
  }.
  {S2s_use_starttls, true}.
  {s2s_certfile, "/path/to/ssl.pem"}.
\end{verbatim}
Note, that for \ind{jabberd 1.4}jabberd 1.4- or \ind{WPJabber}WPJabber-based
services you have to make the transports log and do \ind{XDB}XDB by themselves:
\begin{verbatim}
  <!--
     You have to add elogger and rlogger entries here when using ejabberd.
     In this case the transport will do the logging.
  -->

  <log id='logger'>
    <host/>
    <logtype/>
    <format>%d: [%t] (%h): %s</format>
    <file>/var/log/jabber/service.log</file>
  </log>

  <!--
     Some Jabber server implementations do not provide
     XDB services (for example, jabberd2 and ejabberd).
     xdb_file.so is loaded in to handle all XDB requests.
  -->

  <xdb id="xdb">
    <host/>
    <load>
      <!-- this is a lib of wpjabber or jabberd -->
      <xdb_file>/usr/lib/jabber/xdb_file.so</xdb_file>
      </load>
    <xdb_file xmlns="jabber:config:xdb_file">
      <spool><jabberd:cmdline flag='s'>/var/spool/jabber</jabberd:cmdline></spool>
    </xdb_file>
  </xdb>
\end{verbatim}

\subsection{Authentication}
\label{auth}
\ind{authentication}\ind{options!auth\_method}

The option \option{auth\_method} defines the authentication method that is used
for user authentication:
\begin{verbatim}
  {auth_method, [<method>]}.
\end{verbatim}

The following authentication methods are supported by \ejabberd{}:
\begin{itemize}
\item internal (default) --- See section~\ref{internalauth}.
\item external --- There are \footahref{http://ejabberd.jabber.ru/extauth}{some
 example authentication scripts}.
\item ldap --- See section~\ref{ldap}.
\item odbc --- See section~\ref{mysql}, \ref{pgsql},
  \ref{mssql} and \ref{odbc}.
\item anonymous --- See section~\ref{saslanonymous}.
\end{itemize}

\subsubsection{Internal}
\label{internalauth}
\ind{internal authentication}\ind{Mnesia}

\ejabberd{} uses its internal Mnesia database as the default authentication method.

\begin{itemize}
\item \term{auth\_method}: The value \term{internal} will enable the internal
  authentication method.
\end{itemize}

Examples:
\begin{itemize}
\item To use internal authentication on \jid{example.org} and LDAP
  authentication on \jid{example.net}:
  \begin{verbatim}
{host_config, "example.org", [{auth_method, [internal]}]}.
{host_config, "example.net", [{auth_method, [ldap]}]}.
\end{verbatim}
\item To use internal authentication on all virtual hosts:
  \begin{verbatim}
{auth_method, internal}.
\end{verbatim}
\end{itemize}

If you want to define more \term{registration\_watchers} or \term{modules} for a virtual host and add them to the defined for all the Jabber server, instead of defining the option name simply as \term{modules} use this: \term{\{add, modules\}}.

Examples:
\begin{itemize}
	\item If you defined two global watchers but on a certain virtual host only Ann should be watcher:
\begin{verbatim}
{registration_watchers, ["tom@example.com", "moe@example.net"]}.

{host_config, "example.org", [{registration_watchers, ["ann@example.net"]}]}.
\end{verbatim}

\item If you prefer all three to be registration watchers on that virtual host:
\begin{verbatim}
{registration_watchers, ["tom@example.com", "moe@example.net"]}.

{host_config, "example.org", [{{add, registration_watchers}, ["ann@example.net"]}]}.
\end{verbatim}
\end{itemize}

\subsubsection{SASL Anonymous and Anonymous Login}
\label{saslanonymous}
\ind{sasl anonymous}\ind{anonymous login}

%TODO: introduction; tell what people can do with this
The anonymous authentication method can be configured with the following
options. Remember that you can use the \term{host\_config} option to set virtual
host specific options (see section~\ref{virtualhost}). Note that there also
is a detailed tutorial regarding \footahref{http://support.process-one.net/doc/display/MESSENGER/Anonymous+users+support}{SASL
Anonymous and anonymous login configuration}.

\begin{itemize}
\item \term{auth\_method}: The value \term{anonymous} will enable the anonymous
  authentication method.
\item \term{allow\_multiple\_connections}: This value for this option can be
  either \term{true} or \term{false} and is only used when the anonymous mode is
  enabled. Setting it to \term{true} means that the same username can be taken
  multiple times in anonymous login mode if different resource are used to
  connect. This option is only useful in very special occasions. The default
  value is \term{false}.
\item \term{anonymous\_protocol}: This option can take three values:
  \term{sasl\_anon}, \term{login\_anon} or \term{both}. \term{sasl\_anon} means
  that the SASL Anonymous method will be used. \term{login\_anon} means that the
  anonymous login method will be used. \term{both} means that SASL Anonymous and
  login anonymous are both enabled.
\end{itemize}

Those options are defined for each virtual host with the \term{host\_config}
parameter (see section~\ref{virtualhost}).

Examples:
\begin{itemize}
\item To enable anonymous login on all virtual hosts:
  \begin{verbatim}
{auth_method, [anonymous]}.
{anonymous_protocol, login_anon}.
  \end{verbatim}
\item Similar as previous example, but limited to \jid{public.example.org}:
  \begin{verbatim}
{host_config, "public.example.org", [{auth_method, [anonymous]},
                                     {anonymous_protocol, login_anon}]}.
\end{verbatim}
\item To enable anonymous login and internal authentication on a virtual host:
  \begin{verbatim}
{host_config, "public.example.org", [{auth_method, [internal,anonymous]},
                                     {anonymous_protocol, login_anon}]}.
\end{verbatim}
\item To enable SASL Anonymous on a virtual host:
  \begin{verbatim}
{host_config, "public.example.org", [{auth_method, [anonymous]},
                                     {anonymous_protocol, sasl_anon}]}.
\end{verbatim}
\item To enable SASL Anonymous and anonymous login on a virtual host:
  \begin{verbatim}
{host_config, "public.example.org", [{auth_method, [anonymous]},
                                     {anonymous_protocol, both}]}.
\end{verbatim}
\item To enable SASL Anonymous, anonymous login, and internal authentication on
a virtual host:
  \begin{verbatim}
{host_config, "public.example.org", [{auth_method, [internal,anonymous]},
                                     {anonymous_protocol, both}]}.
\end{verbatim}
\end{itemize}

\subsection{Access Rules}
\label{accessrules}
\ind{access rules}\ind{ACL}\ind{Access Control List}

\subsubsection{ACL Definition}
\label{ACLDefinition}
\ind{ACL}\ind{options!acl}\ind{ACL}\ind{Access Control List}

Access control in \ejabberd{} is performed via Access Control Lists (ACLs). The
declarations of ACLs in the configuration file have the following syntax:
\begin{verbatim}
  {acl, <aclname>, {<acltype>, ...}}.
\end{verbatim}
\term{<acltype>} can be one of the following:
\begin{description}
\titem{all} Matches all JIDs. Example:
\begin{verbatim}
{acl, all, all}.
\end{verbatim}
\titem{\{user, <username>\}} Matches the user with the name
  \term{<username>} at the first virtual host. Example:
\begin{verbatim}
{acl, admin, {user, "yozhik"}}.
\end{verbatim}
\titem{\{user, <username>, <server>\}} Matches the user with the JID
  \term{<username>@<server>} and any resource. Example:
\begin{verbatim}
{acl, admin, {user, "yozhik", "example.org"}}.
\end{verbatim}
\titem{\{server, <server>\}} Matches any JID from server
  \term{<server>}. Example:
\begin{verbatim}
{acl, exampleorg, {server, "example.org"}}.
\end{verbatim}
\titem{\{user\_regexp, <regexp>\}} Matches any local user with a name that
  matches \term{<regexp>} at the first virtual host. Example:
\begin{verbatim}
{acl, tests, {user_regexp, "^test[0-9]*$"}}.
\end{verbatim}
%$
\titem{\{user\_regexp, <regexp>, <server>\}} Matches any user with a name
  that matches \term{<regexp>} at server \term{<server>}. Example:
\begin{verbatim}
{acl, tests, {user_regexp, "^test", "example.org"}}.
\end{verbatim}
\titem{\{server\_regexp, <regexp>\}} Matches any JID from the server that
  matches \term{<regexp>}. Example:
\begin{verbatim}
{acl, icq, {server_regexp, "^icq\\."}}.
\end{verbatim}
\titem{\{node\_regexp, <user\_regexp>, <server\_regexp>\}} Matches any user
  with a name that matches \term{<user\_regexp>} at any server that matches
  \term{<server\_regexp>}. Example:
\begin{verbatim}
{acl, yohzik, {node_regexp, "^yohzik$", "^example.(com|org)$"}}.
\end{verbatim}
\titem{\{user\_glob, <glob>\}}
\titem{\{user\_glob, <glob>, <server>\}}
\titem{\{server\_glob, <glob>\}}
\titem{\{node\_glob, <user\_glob>, <server\_glob>\}} This is the same as
  above. However, it uses shell glob patterns instead of regexp. These patterns
  can have the following special characters:
  \begin{description}
  \titem{*} matches any string including the null string.
  \titem{?} matches any single character.
  \titem{[...]} matches any of the enclosed characters. Character
    ranges are specified by a pair of characters separated by a \term{`-'}.
    If the first character after \term{`['} is a \term{`!'}, any
    character not enclosed is matched.
  \end{description}
\end{description}

The following ACLs are pre-defined:
\begin{description}
\titem{all} Matches any JID.
\titem{none} Matches no JID.
\end{description}

\subsubsection{Access Rights}
\label{AccessRights}
\ind{access}\ind{ACL}\ind{options!acl}\ind{ACL}\ind{Access Control List}

An entry allowing or denying access to different services looks similar to
this:
\begin{verbatim}
  {access, <accessname>, [{allow, <aclname>},
                          {deny, <aclname>},
                          ...
                         ]}.
\end{verbatim}
When a JID is checked to have access to \term{<accessname>}, the server
sequentially checks if that JID mathes any of the ACLs that are named in the
second elements of the tuples in the list. If it matches, the first element of
the first matched tuple is returned, otherwise the value `\term{deny}' is
returned.

Example:
\begin{verbatim}
  {access, configure, [{allow, admin}]}.
  {access, something, [{deny, badmans},
                       {allow, all}]}.
\end{verbatim}

The following access rules are pre-defined:
\begin{description}
\titem{all} Always returns the value `\term{allow}'.
\titem{none} Always returns the value `\term{deny}'.
\end{description}

\subsubsection{Limiting Opened Sessions with ACL}
\label{configmaxsessions}
\ind{options!max\_user\_sessions}

The special access \term{max\_user\_sessions} specifies the maximum number of sessions (authenticated
connections) per user. If a user tries to open more sessions by using different
resources, the first opened session will be disconnected. The error
\term{session replaced} will be sent to the disconnected session. The value
for this option can be either a number, or \term{infinity}.  The default
value is \term{infinity}.

The syntax is:
\begin{verbatim}
  {access, max_user_sessions, [{<maxnumber>, <aclname>},
                               ...
                              ]}.
\end{verbatim}

Examples:
\begin{itemize}
\item To limit the number of sessions per user to 10 for all users:
\begin{verbatim}
  {access, max_user_sessions, [{10, all}]}.
\end{verbatim}
\end{itemize}

\subsection{Shapers}
\label{shapers}
\ind{options!shaper}\ind{options!maxrate}\ind{shapers}\ind{maxrate}\ind{traffic speed}

Shapers enable you to limit connection traffic. The syntax of
shapers is like this:
\begin{verbatim}
  {shaper, <shapername>, <kind>}.
\end{verbatim}
Currently only one kind of shaper called \term{maxrate} is available. It has the
following syntax:
\begin{verbatim}
  {maxrate, <rate>}
\end{verbatim}
where \term{<rate>} stands for the maximum allowed incomig rate in bytes per
second.

Examples:
\begin{itemize}
\item To define a shaper named `\term{normal}' with traffic speed limited to
1,000\,bytes/second:
\begin{verbatim}
  {shaper, normal, {maxrate, 1000}}.
\end{verbatim}
\item To define a shaper named `\term{fast}' with traffic speed limited to
50,000\,bytes/second:
\begin{verbatim}
  {shaper, fast, {maxrate, 50000}}.
\end{verbatim}
\end{itemize}

\subsection{Default Language}
\label{language}
\ind{options!language}\ind{language}

The option \option{language} defines the default language of server strings that
can be seen by \Jabber{} clients. If a \Jabber{} client do not support
\option{xml:lang}, the specified language is used. The default value is
\term{en}. In order to take effect there must be a translation file
\term{<language>.msg} in \ejabberd{}'s \term{msgs} directory.

Examples:
\begin{itemize}
\item To set Russian as default language:
\begin{verbatim}
  {language, "ru"}.
\end{verbatim}
\item To set Spanish as default language:
\begin{verbatim}
  {language, "es"}.
\end{verbatim}
\end{itemize}

\section{Database and LDAP Configuration}
\label{database}
\ind{database}
%TODO: this whole section is not yet 100% optimized

\ejabberd{} uses its internal Mnesia database by default. However, it is
possible to use a relational database or an LDAP server to store persistant,
long-living data. \ejabberd{} is very flexible: you can configure different
authentication methods for different virtual hosts, you can configure different
authentication mechanisms for the same virtual host (fallback), you can set
different storage systems for modules, and so forth.

The following databases are supported by \ejabberd{}:
\begin{itemize}
\item \footahref{http://www.microsoft.com/sql/}{Microsoft SQL Server}
\item \footahref{http://www.erlang.org/doc/doc-5.5.1/lib/mnesia-4.3.2/doc/}{Mnesia}
\item \footahref{http://mysql.com/}{MySQL}
\item \footahref{http://en.wikipedia.org/wiki/Open\_Database\_Connectivity}{Any ODBC compatible database}
\item \footahref{http://www.postgresql.org/}{PostgreSQL}
\end{itemize}

The following LDAP servers are tested with \ejabberd{}:
\begin{itemize}
\item \footahref{http://www.microsoft.com/activedirectory/}{Active Directory}
  (see section~\ref{ad})
\item \footahref{http://www.openldap.org/}{OpenLDAP}
\item Normally any LDAP compatible server should work; inform us about your
  success with a not-listed server so that we can list it here.
\end{itemize}

\subsection{MySQL}
\label{mysql}
\ind{MySQL}\ind{MySQL!schema}

Although this section will describe \ejabberd{}'s configuration when you want to
use the native MySQL driver, it does not describe MySQL's installation and
database creation. Check the MySQL documentation and the tutorial \footahref{http://support.process-one.net/doc/display/MESSENGER/Using+ejabberd+with+MySQL+native+driver}{Using ejabberd with MySQL native driver} for information regarding these topics.
Note that the tutorial contains information about \ejabberd{}'s configuration
which is duplicate to this section.

Moreover, the file mysql.sql in the directory src/odbc might be interesting for
you. This file contains the ejabberd schema for MySQL. At the end of the file
you can find information to update your database schema.

\subsubsection{Driver Compilation}
\label{compilemysql}
\ind{MySQL!Driver Compilation}

You can skip this step if you installed \ejabberd{} using a binary installer or
if the binary packages of \ejabberd{} you are using include support for MySQL.

\begin{enumerate}
\item First, install the \footahref{http://support.process-one.net/doc/display/CONTRIBS/Yxa}{Erlang
  MySQL library}. Make sure the compiled files are in your Erlang path; you can
  put them for example in the same directory as your ejabberd .beam files.
\item Then, configure and install \ejabberd{} with ODBC support enabled (this is
  also needed for native MySQL support!). This can be done, by using next
  commands:
  \begin{verbatim}
./configure --enable-odbc && make install
\end{verbatim}
\end{enumerate}

\subsubsection{Authentication}
\label{mysqlauth}
\ind{MySQL!authentication}

The option value name may be misleading, as the \term{auth\_method} name is used
for access to a relational database through ODBC, as well as through the native
MySQL interface. Anyway, the first configuration step is to define the odbc
\term{auth\_method}. For example:
\begin{verbatim}
{host_config, "public.example.org", [{auth_method, [odbc]}]}.
\end{verbatim}

The actual database access is defined in the option \term{odbc\_server}. Its
value is used to define if we want to use ODBC, or one of the two native
interface available, PostgreSQL or MySQL.

To use the native MySQL interface, you can pass a tuple of the following form as
parameter:
\begin{verbatim}
{mysql, "Server", "Database", "Username", "Password"}
\end{verbatim}

\term{mysql} is a keyword that should be kept as is. For example:
\begin{verbatim}
{odbc_server, {mysql, "localhost", "test", "root", "password"}}.
\end{verbatim}

Optionally, it is possible to define the MySQL port to use. This
option is only useful, in very rare cases, when you are not running
MySQL with the default port setting. The \term{mysql} parameter
can thus take the following form:
\begin{verbatim}
{mysql, "Server", Port, "Database", "Username", "Password"}
\end{verbatim}

The \term{Port} value should be an integer, without quotes. For example:
\begin{verbatim}
{odbc_server, {mysql, "localhost", Port, "test", "root", "password"}}.
\end{verbatim}


\subsubsection{Storage}
\label{mysqlstorage}
\ind{MySQL!storage}

MySQL also can be used to store information into from several \ejabberd{}
modules. See section~\ref{modoverview} to see which modules have a version
with the `\_odbc'. This suffix indicates that the module can be used with
relational databases like MySQL. To enable storage to your database, just make
sure that your database is running well (see previous sections), and replace the
suffix-less or ldap module variant with the odbc module variant. Keep in mind
that you cannot have several variants of the same module loaded!

\subsection{Microsoft SQL Server}
\label{mssql}
\ind{Microsoft SQL Server}\ind{Microsoft SQL Server!schema}

Although this section will describe \ejabberd{}'s configuration when you want to
use Microsoft SQL Server, it does not describe Microsoft SQL Server's
installation and database creation. Check the MySQL documentation and the
tutorial \footahref{http://support.process-one.net/doc/display/MESSENGER/Using+ejabberd+with+MySQL+native+driver}{Using ejabberd with MySQL native driver} for information regarding these topics.
Note that the tutorial contains information about \ejabberd{}'s configuration
which is duplicate to this section.

Moreover, the file mssql.sql in the directory src/odbc might be interesting for
you. This file contains the ejabberd schema for Microsoft SQL Server. At the end
of the file you can find information to update your database schema.

\subsubsection{Driver Compilation}
\label{compilemssql}
\ind{Microsoft SQL Server!Driver Compilation}

You can skip this step if you installed \ejabberd{} using a binary installer or
if the binary packages of \ejabberd{} you are using include support for ODBC.

If you want to use Microsoft SQL Server with ODBC, you need to configure,
compile and install \ejabberd{} with support for ODBC and Microsoft SQL Server
enabled. This can be done, by using next commands:
\begin{verbatim}
./configure --enable-odbc --enable-mssql && make install
\end{verbatim}

\subsubsection{Authentication}
\label{mssqlauth}
\ind{Microsoft SQL Server!authentication}

%TODO: not sure if this section is right!!!!!!

The configuration of Microsoft SQL Server is the same as the configuration of
ODBC compatible serers (see section~\ref{odbcauth}).

\subsubsection{Storage}
\label{mssqlstorage}
\ind{Microsoft SQL Server!storage}

Microsoft SQL Server also can be used to store information into from several
\ejabberd{} modules. See section~\ref{modoverview} to see which modules have
a version with the `\_odbc'. This suffix indicates that the module can be used
with relational databases like Microsoft SQL Server. To enable storage to your
database, just make sure that your database is running well (see previous
sections), and replace the suffix-less or ldap module variant with the odbc
module variant. Keep in mind that you cannot have several variants of the same
module loaded!

\subsection{PostgreSQL}
\label{pgsql}
\ind{PostgreSQL}\ind{PostgreSQL!schema}

Although this section will describe \ejabberd{}'s configuration when you want to
use the native PostgreSQL driver, it does not describe PostgreSQL's installation
and database creation. Check the PostgreSQL documentation and the tutorial \footahref{http://support.process-one.net/doc/display/MESSENGER/Using+ejabberd+with+MySQL+native+driver}{Using ejabberd with MySQL native driver} for information regarding these topics.
Note that the tutorial contains information about \ejabberd{}'s configuration
which is duplicate to this section.

Also the file pg.sql in the directory src/odbc might be interesting for you.
This file contains the ejabberd schema for PostgreSQL. At the end of the file
you can find information to update your database schema.

\subsubsection{Driver Compilation}
\label{compilepgsql}
\ind{PostgreSQL!Driver Compilation}

You can skip this step if you installed \ejabberd{} using a binary installer or
if the binary packages of \ejabberd{} you are using include support for
PostgreSQL.

\begin{enumerate}
\item First, install the Erlang PgSQL library from
  \footahref{http://jungerl.sourceforge.net/}{Jungerl}. Make sure the compiled
  files are in your Erlang path; you can put them for example in the same
  directory as your ejabberd .beam files.
\item Then, configure, compile and install \ejabberd{} with ODBC support enabled
  (this is also needed for native PostgreSQL support!). This can be done, by
  using next commands:
  \begin{verbatim}
./configure --enable-odbc && make install
\end{verbatim}
\end{enumerate}

\subsubsection{Authentication}
\label{pgsqlauth}
\ind{PostgreSQL!authentication}

The option value name may be misleading, as the \term{auth\_method} name is used
for access to a relational database through ODBC, as well as through the native
PostgreSQL interface. Anyway, the first configuration step is to define the odbc
\term{auth\_method}. For example:
\begin{verbatim}
{host_config, "public.example.org", [{auth_method, [odbc]}]}.
\end{verbatim}

The actual database access is defined in the option \term{odbc\_server}. Its
value is used to define if we want to use ODBC, or one of the two native
interface available, PostgreSQL or MySQL.

To use the native PostgreSQL interface, you can pass a tuple of the following
form as parameter:
\begin{verbatim}
{pgsql, "Server", "Database", "Username", "Password"}
\end{verbatim}

\term{pgsql} is a keyword that should be kept as is. For example:
\begin{verbatim}
{odbc_server, {pgsql, "localhost", "database", "ejabberd", "password"}}.
\end{verbatim}

Optionally, it is possible to define the PostgreSQL port to use. This
option is only useful, in very rare cases, when you are not running
PostgreSQL with the default port setting. The \term{pgsql} parameter
can thus take the following form:
\begin{verbatim}
{pgsql, "Server", Port, "Database", "Username", "Password"}
\end{verbatim}

The \term{Port} value should be an integer, without quotes. For example:
\begin{verbatim}
{odbc_server, {pgsql, "localhost", 5432, "database", "ejabberd", "password"}}.
\end{verbatim}

\subsubsection{Storage}
\label{pgsqlstorage}
\ind{PostgreSQL!storage}

PostgreSQL also can be used to store information into from several \ejabberd{}
modules. See section~\ref{modoverview} to see which modules have a version
with the `\_odbc'. This suffix indicates that the module can be used with
relational databases like PostgreSQL. To enable storage to your database, just
make sure that your database is running well (see previous sections), and
replace the suffix-less or ldap module variant with the odbc module variant.
Keep in mind that you cannot have several variants of the same module loaded!

\subsection{ODBC Compatible}
\label{odbc}
\ind{databases!ODBC}

Although this section will describe \ejabberd{}'s configuration when you want to
use the ODBC driver, it does not describe the installation and database creation
of your database. Check the documentation of your database. The tutorial \footahref{http://support.process-one.net/doc/display/MESSENGER/Using+ejabberd+with+MySQL+native+driver}{Using ejabberd with MySQL native driver} also can help you. Note that the tutorial
contains information about \ejabberd{}'s configuration which is duplicate to
this section.

\subsubsection{Compilation}
\label{compileodbc}

You can skip this step if you installed \ejabberd{} using a binary installer or
if the binary packages of \ejabberd{} you are using include support for
ODBC.

\begin{enumerate}
\item First, install the \footahref{http://support.process-one.net/doc/display/CONTRIBS/Yxa}{Erlang
  MySQL library}. Make sure the compiled files are in your Erlang path; you can
  put them for example in the same directory as your ejabberd .beam files.
\item Then, configure, compile and install \ejabberd{} with ODBC support
  enabled. This can be done, by using next commands:
  \begin{verbatim}
./configure --enable-odbc && make install
\end{verbatim}
\end{enumerate}

\subsubsection{Authentication}
\label{odbcauth}
\ind{ODBC!authentication}

The first configuration step is to define the odbc \term{auth\_method}. For
example:
\begin{verbatim}
{host_config, "public.example.org", [{auth_method, [odbc]}]}.
\end{verbatim}

The actual database access is defined in the option \term{odbc\_server}. Its
value is used to defined if we want to use ODBC, or one of the two native
interface available, PostgreSQL or MySQL.

To use a relational database through ODBC, you can pass the ODBC connection
string as \term{odbc\_server} parameter. For example:
\begin{verbatim}
{odbc_server, "DSN=database;UID=ejabberd;PWD=password"}.
\end{verbatim}

\subsubsection{Storage}
\label{odbcstorage}
\ind{ODBC!storage}

An ODBC compatible database also can be used to store information into from
several \ejabberd{} modules. See section~\ref{modoverview} to see which
modules have a version with the `\_odbc'. This suffix indicates that the module
can be used with ODBC compatible relational databases. To enable storage to your
database, just make sure that your database is running well (see previous
sections), and replace the suffix-less or ldap module variant with the odbc
module variant. Keep in mind that you cannot have several variants of the same
module loaded!

\subsection{LDAP}
\label{ldap}
\ind{databases!LDAP}

\ejabberd{} has built-in LDAP support. You can authenticate users against LDAP
server and use LDAP directory as vCard storage. Shared rosters are not supported
yet.

\subsubsection{Connection}
\label{ldapconnection}

Parameters:
\begin{description}
\titem{ldap\_server} \ind{options!ldap\_server}IP address or dns name of your
LDAP server. This option is required.
\titem{ldap\_port} \ind{options!ldap\_port}Port to connect to your LDAP server.
  The default value is~389.
\titem{ldap\_rootdn} \ind{options!ldap\_rootdn}Bind DN. The default value
  is~\term{""} which means `anonymous connection'.
\titem{ldap\_password} \ind{options!ldap\_password}Bind password. The default
  value is \term{""}.
\end{description}

Example:
\begin{verbatim}
  {auth_method, ldap}.
  {ldap_servers, ["ldap.example.org"]}.
  {ldap_port, 389}.
  {ldap_rootdn, "cn=Manager,dc=domain,dc=org"}.
  {ldap_password, "secret"}.
\end{verbatim}

Note that current LDAP implementation does not support SSL secured communication
and SASL authentication.

\subsubsection{Authentication}
\label{ldapauth}

You can authenticate users against an LDAP directory. Available options are:

\begin{description}
\titem{ldap\_base}\ind{options!ldap\_base}LDAP base directory which stores
  users accounts. This option is required.
  \titem{ldap\_uids}\ind{options!ldap\_uids}LDAP attribute which holds a list
  of attributes to use as alternatives for getting the JID. The value is of
  the form: \term{[\{ldap\_uidattr\}]} or \term{[\{ldap\_uidattr,
  ldap\_uidattr\_format\}]}. You can use as many comma separated tuples
  \term{\{ldap\_uidattr, ldap\_uidattr\_format\}} that is needed. The default
  value is \term{[\{"uid", "\%u"\}]}. The defaut \term{ldap\_uidattr\_format}
  is \term{"\%u"}. The values for \term{ldap\_uidattr} and
  \term{ldap\_uidattr\_format} are described as follow:
  \begin{description}
    \titem{ldap\_uidattr}\ind{options!ldap\_uidattr}LDAP attribute which holds
    the user's part of a JID. The default value is \term{"uid"}.
    \titem{ldap\_uidattr\_format}\ind{options!ldap\_uidattr\_format}Format of
    the \term{ldap\_uidattr} variable. The format \emph{must} contain one and
    only one pattern variable \term{"\%u"} which will be replaced by the
    user's part of a JID. For example, \term{"\%u@example.org"}. The default
    value is \term{"\%u"}.
  \end{description}
  \titem{ldap\_filter}\ind{options!ldap\_filter}\ind{protocols!RFC 2254: The
  String Representation of LDAP Search Filters}
  \footahref{http://www.faqs.org/rfcs/rfc2254.html}{RFC 2254} LDAP filter. The
  default is \term{none}. Example:
  \term{"(\&(objectClass=shadowAccount)(memberOf=Jabber Users))"}. Please, do
  not forget to close brackets and do not use superfluous whitespaces. Also you
  \emph{must not} use \option{ldap\_uidattr} attribute in filter because this
  attribute will be substituted in LDAP filter automatically.
\end{description}

\subsubsection{Examples}
\label{ldapexamples}

\paragraph{\aname{ldapcommonexample}{Common example}}

Let's say \term{ldap.example.org} is the name of our LDAP server. We have
users with their passwords in \term{"ou=Users,dc=example,dc=org"} directory.
Also we have addressbook, which contains users emails and their additional
infos in \term{"ou=AddressBook,dc=example,dc=org"} directory.  Corresponding
authentication section should looks like this:

\begin{verbatim}
  %% authentication method
  {auth_method, ldap}.
  %% DNS name of our LDAP server
  {ldap_servers, ["ldap.example.org"]}.
  %% Bind to LDAP server as "cn=Manager,dc=example,dc=org" with password "secret"
  {ldap_rootdn, "cn=Manager,dc=example,dc=org"}.
  {ldap_password, "secret"}.
  %% define the user's base
  {ldap_base, "ou=Users,dc=example,dc=org"}.
  %% We want to authorize users from 'shadowAccount' object class only
  {ldap_filter, "(objectClass=shadowAccount)"}.
\end{verbatim}

Now we want to use users LDAP-info as their vCards.  We have four attributes
defined in our LDAP schema: \term{"mail"} --- email address, \term{"givenName"}
--- first name, \term{"sn"} --- second name, \term{"birthDay"} --- birthday.
Also we want users to search each other.  Let's see how we can set it up:

\begin{verbatim}
  {modules,
    ...
    {mod_vcard_ldap,
     [
      %% We use the same server and port, but want to bind anonymously because
      %% our LDAP server accepts anonymous requests to
      %% "ou=AddressBook,dc=example,dc=org" subtree.
      {ldap_rootdn, ""},
      {ldap_password, ""},
      %% define the addressbook's base
      {ldap_base, "ou=AddressBook,dc=example,dc=org"},
      %% uidattr: user's part of JID is located in the "mail" attribute
      %% uidattr_format: common format for our emails
      {ldap_uids, [{"mail", "%u@mail.example.org"}]},
      %% We have to define empty filter here, because entries in addressbook does not
      %% belong to shadowAccount object class
      {ldap_filter, ""},
      %% Now we want to define vCard pattern
      {ldap_vcard_map,
       [{"NICKNAME", "%u", []}, % just use user's part of JID as his nickname
        {"GIVEN", "%s", ["givenName"]},
        {"FAMILY", "%s", ["sn"]},
        {"FN", "%s, %s", ["sn", "givenName"]}, % example: "Smith, John"
        {"EMAIL", "%s", ["mail"]},
        {"BDAY", "%s", ["birthDay"]}]},
      %% Search form
      {ldap_search_fields,
       [{"User", "%u"},
        {"Name", "givenName"},
        {"Family Name", "sn"},
        {"Email", "mail"},
        {"Birthday", "birthDay"}]},
      %% vCard fields to be reported
      %% Note that JID is always returned with search results
      {ldap_search_reported,
       [{"Full Name", "FN"},
        {"Nickname", "NICKNAME"},
        {"Birthday", "BDAY"}]}
    ]}
    ...
  }.
\end{verbatim}

Note that \modvcardldap{} module checks for the existence of the user before
searching in his information in LDAP.


\paragraph{Active Directory}
\label{ad}
\ind{databases!Active Directory}

Active Directory is just an LDAP-server with predefined attributes. A sample
configuration is showed below:

\begin{verbatim}
  {auth_method, ldap}.
  {ldap_servers, ["office.org"]}.    % List of LDAP servers
  {ldap_base, "DC=office,DC=org"}. % Search base of LDAP directory
  {ldap_rootdn, "CN=Administrator,CN=Users,DC=office,DC=org"}. % LDAP manager
  {ldap_password, "*******"}. % Password to LDAP manager
  {ldap_uids, [{"sAMAccountName"}]}.
  {ldap_filter, "(memberOf=*)"}.

  {mod_vcard_ldap,
   [{ldap_vcard_map,
     [{"NICKNAME", "%u", []},
      {"GIVEN", "%s", ["givenName"]},
      {"MIDDLE", "%s", ["initials"]},
      {"FAMILY", "%s", ["sn"]},
      {"FN", "%s", ["displayName"]},
      {"EMAIL", "%s", ["mail"]},
      {"ORGNAME", "%s", ["company"]},
      {"ORGUNIT", "%s", ["department"]},
      {"CTRY", "%s", ["c"]},
      {"LOCALITY", "%s", ["l"]},
      {"STREET", "%s", ["streetAddress"]},
      {"REGION", "%s", ["st"]},
      {"PCODE", "%s", ["postalCode"]},
      {"TITLE", "%s", ["title"]},
      {"URL", "%s", ["wWWHomePage"]},
      {"DESC", "%s", ["description"]},
      {"TEL", "%s", ["telephoneNumber"]}]},
    {ldap_search_fields,
     [{"User", "%u"},
      {"Name", "givenName"},
      {"Family Name", "sn"},
      {"Email", "mail"},
      {"Company", "company"},
      {"Department", "department"},
      {"Role", "title"},
      {"Description", "description"},
      {"Phone", "telephoneNumber"}]},
    {ldap_search_reported,
     [{"Full Name", "FN"},
      {"Nickname", "NICKNAME"},
      {"Email", "EMAIL"}]}
   ]
  }.
\end{verbatim}


\section{Modules Configuration}
\label{modules}
\ind{modules}

The option \term{modules} defines the list of modules that will be loaded after
\ejabberd{}'s startup. Each entry in the list is a tuple in which the first
element is the name of a module and the second is a list of options for that
module.

Examples:
\begin{itemize}
\item In this example only the module \modecho{} is loaded and no module
  options are specified between the square brackets:
  \begin{verbatim}
  {modules,
   [{mod_echo,      []}
   ]}.
\end{verbatim}
\item In the second example the modules \modecho{}, \modtime{}, and
  \modversion{} are loaded without options. Remark that, besides the last entry,
  all entries end with a comma:
  \begin{verbatim}
  {modules,
   [{mod_echo,      []},
    {mod_time,      []},
    {mod_version,   []}
   ]}.
\end{verbatim}
\end{itemize}

\subsection{Overview}
\label{modoverview}
\ind{modules!overview}\ind{XMPP compliancy}

The following table lists all modules available in the official \ejabberd{}
distribution. You can find more
\footahref{http://ejabberd.jabber.ru/contributions}{contributed modules} on the
\ejabberd{} website. Please remember that these contributions might not work or
that they can contain severe bugs and security leaks. Therefore, use them at
your own risk!

You can see which database backend each module needs by looking at the suffix:
\begin{itemize}
\item `\_ldap', this means that the module needs an LDAP server as backend.
\item `\_odbc', this means that the module needs a supported database
  (see~\ref{database}) as backend.
\item No suffix, this means that the modules uses Erlang's built-in database
  Mnesia as backend.
\end{itemize}

If you want to
It is possible to use a relational database to store pieces of
information. You can do this by changing the module name to a name with an
\term{\_odbc} suffix in \ejabberd{} config file. You can use a relational
database for the following data:

\begin{itemize}
\item Last connection date and time: Use \term{mod\_last\_odbc} instead of
  \term{mod\_last}.
\item Offline messages: Use \term{mod\_offline\_odbc} instead of
  \term{mod\_offline}.
\item Rosters: Use \term{mod\_roster\_odbc} instead of \term{mod\_roster}.
\item Users' VCARD: Use \term{mod\_vcard\_odbc} instead of \term{mod\_vcard}.
\end{itemize}


\begin{table}[H]
  \centering
  \begin{tabular}{|l|l|l|l|}
    \hline Module & Feature & Dependencies & Needed for XMPP? \\ 
    \hline \hline \modadhoc{} & Ad-Hoc Commands (\xepref{0050}) &  & No \\ 
    \hline \modannounce{} & Manage announcements & \modadhoc{} & No \\ 
    \hline \modconfigure{} & Support for online & \modadhoc{} & No \\ 
    & configuration of \ejabberd{} & & \\ 
    \hline \moddisco{} & Service Discovery (\xepref{0030}) &  & No \\ 
    \hline \modecho{} & Echoes Jabber packets &  & No \\ 
    \hline \modirc{} & IRC transport &  & No \\ 
    \hline \modlast{} & Last Activity (\xepref{0012}) &  & No \\ 
    \hline \modlastodbc{} & Last Activity (\xepref{0012}) & supported database (*) & No \\ 
    \hline \modmuc{} & Multi-User Chat (\xepref{0045}) &  & No \\ 
    \hline \modmuclog{} & Multi-User Chat room logging & \modmuc{} & No \\ 
    \hline \modoffline{} & Offline message storage &  & No \\ 
    \hline \modofflineodbc{} & Offline message storage & supported database (*) & No \\ 
    \hline \modprivacy{} & Blocking Communication &  & Yes \\ 
    \hline \modprivate{} & Private XML Storage (\xepref{0049}) &  & No \\ 
    \hline \modprivateodbc{} & Private XML Storage (\xepref{0049}) & supported database (*) & No \\ 
    \hline \modproxy{} & SOCKS5 Bytestreams (\xepref{0065}) &  & No\\
    \hline \modpubsub{} & Publish-Subscribe (\xepref{0060}) &  & No \\ 
    \hline \modregister{} & In-Band Registration (\xepref{0077}) &  & No \\ 
    \hline \modroster{} & Roster management &  & Yes (**) \\ 
    \hline \modrosterodbc{} & Roster management & supported database (*) & Yes (**) \\ 
    \hline \modservicelog{} & Copy user messages to logger service &  & No \\ 
    \hline \modsharedroster{} & Shared roster management & \modroster{} or & No \\ 
    & & \modrosterodbc{} & \\ 
    \hline \modstats{} & Statistics Gathering (\xepref{0039}) &  & No \\ 
    \hline \modtime{} & Entity Time (\xepref{0090}) &  & No \\ 
    \hline \modvcard{} & vcard-temp (\xepref{0054}) &  & No \\ 
    \hline \modvcardldap{} & vcard-temp (\xepref{0054}) & LDAP server & No \\ 
    \hline \modvcardodbc{} & vcard-temp (\xepref{0054}) & supported database (*) & No \\ 
    \hline \modversion{} & Software Version (\xepref{0092}) &  & No\\
    \hline
  \end{tabular}
\end{table}

\begin{itemize}
\item (*) For a list of supported databases, see section~\ref{database}.
\item (**) This module or a similar one with another database backend is needed for
XMPP compliancy.
\end{itemize}

\subsection{Common Options}
\label{modcommonoptions}

The following options are used by many modules. Therefore, they are described in
this separate section.

\subsubsection{\option{iqdisc}}
\label{modiqdiscoption}
\ind{options!iqdisc}

Many modules define handlers for processing IQ queries of different namespaces
to this server or to a user (e.\,g.\ to \jid{example.org} or to
\jid{user@example.org}). This option defines processing discipline for
these queries. Possible values are:
\begin{description}
\titem{no\_queue} All queries of a namespace with this processing discipline are
  processed immediately. This also means that no other packets can be processed
  until this one has been completely processed. Hence this discipline is not
  recommended if the processing of a query can take a relatively long time.
\titem{one\_queue} In this case a separate queue is created for the processing
  of IQ queries of a namespace with this discipline. In addition, the processing
  of this queue is done in parallel with that of other packets. This discipline
  is most recommended.
\titem{parallel} For every packet with this discipline a separate Erlang process
  is spawned. Consequently, all these packets are processed in parallel.
  Although spawning of Erlang process has a relatively low cost, this can break
  the server's normal work, because the Erlang emulator has a limit on the
  number of processes (32000 by default).
\end{description}

Example:
\begin{verbatim}
  {modules,
   [
    ...
    {mod_time, [{iqdisc, no_queue}]},
    ...
   ]}.
\end{verbatim}

\subsubsection{\option{hosts}}
\label{modhostsoption}
\ind{options!hosts}

A module acting as a service can have one or more hostnames. These hostnames
can be defined with the \option{hosts} option.

Examples:
\begin{itemize}
\item Serving the \ind{modules!\modecho{}}echo module on one domain:
  \begin{itemize}
  \item
    \begin{verbatim}
  {modules,
   [
    ...
    {mod_echo, [{hosts, ["echo.example.org"]}]},
    ...
   ]}.
\end{verbatim}
  \item Backwards compatibility with older \ejabberd{} versions can be retained
    with:
    \begin{verbatim}
  {modules,
   [
    ...
    {mod_echo, [{host, "echo.example.org"}]},
    ...
   ]}.
\end{verbatim}
  \end{itemize}
  \item Serving the echo module on two domains:
    \begin{verbatim}
  {modules,
   [
    ...
    {mod_echo, [{hosts, ["echo.example.net", "echo.example.com"]}]},
    ...
   ]}.
\end{verbatim}
\end{itemize}

\subsection{\modannounce{}}
\label{modannounce}
\ind{modules!\modannounce{}}\ind{MOTD}\ind{message of the day}\ind{announcements}

This module enables configured users to broadcast announcements and to set
the message of the day (MOTD). Configured users can do these actions with their
\Jabber{} client using Ad-hoc commands or by sending messages to specific JIDs. These JIDs are listed in
next paragraph. The first JID in each entry will apply only to the virtual host
\jid{example.org}, while the JID between brackets will apply to all virtual
hosts:
\begin{description}
\titem{example.org/announce/all (example.org/announce/all-hosts/all)} The
  message is sent to all registered users. If the user is online and connected
  to several resources, only the resource with the highest priority will receive
  the message. If the registered user is not connected, the message will be
  stored offline in assumption that \ind{modules!\modoffline{}}offline storage
  (see section~\ref{modoffline}) is enabled.
\titem{example.org/announce/online (example.org/announce/all-hosts/online)}The
  message is sent to all connected users. If the user is online and connected
  to several resources, all resources will receive the message.
\titem{example.org/announce/motd (example.org/announce/all-hosts/motd)}The
  message is set as the message of the day (MOTD) and is sent to users when they
  login. In addition the message is sent to all connected users (similar to
  \term{announce/online}).
\titem{example.org/announce/motd/update (example.org/announce/all-hosts/motd/update)}
  The message is set as message of the day (MOTD) and is sent to users when they
  login. The message is \emph{not sent} to any currently connected user.
\titem{example.org/announce/motd/delete (example.org/announce/all-hosts/motd/delete)}
  Any message sent to this JID removes the existing message of the day (MOTD).
\end{description}

Options:
\begin{description}
\titem{access} \ind{options!access}This option specifies who is allowed to
  send announcements and to set the message of the day (by default, nobody is
  able to send such messages).
\end{description}

Examples:
\begin{itemize}
\item Only administrators can send announcements:
  \begin{verbatim}
  {access, announce, [{allow, admins}]}.

  {modules,
   [
    ...
    {mod_announce, [{access, announce}]},
    ...
   ]}.
\end{verbatim}
\item Administrators as well as the direction can send announcements:
\begin{verbatim}
  {acl, direction, {user, "big_boss", "example.org"}}.
  {acl, direction, {user, "assistant", "example.org"}}.
  {acl, admins, {user, "admin", "example.org"}}.
  ...
  {access, announce, [{allow, admins},
                      {allow, direction}]}.
  ...
  {modules,
   [
    ...
    {mod_announce, [{access, announce}]},
    ...
   ]}.
\end{verbatim}
\end{itemize}

Note that \modannounce{} can be resource intensive on large
deployments as it can broadcast lot of messages. This module should be
disabled for instances of ejabberd with hundreds of thousands users.

\subsection{\moddisco{}}
\label{moddisco}
\ind{modules!\moddisco{}}\ind{protocols!XEP-0030: Service Discovery}\ind{protocols!XEP-0011: Jabber Browsing}\ind{protocols!XEP-0094: Agent Information}

This module adds support for Service Discovery (\xepref{0030}). With
this module enabled, services on your server can be discovered by
\Jabber{} clients. Note that \ejabberd{} has no modules with support
for the superseded Jabber Browsing (\xepref{0011}) and Agent Information
(\xepref{0094}). Accordingly, \Jabber{} clients need to have support for
the newer Service Discovery protocol if you want them be able to discover
the services you offer.

Options:
\begin{description}
\iqdiscitem{Service Discovery (\ns{http://jabber.org/protocol/disco\#items} and
  \ns{http://jabber.org/protocol/disco\#info})}
\titem{extra\_domains} \ind{options!extra\_domains}With this option,
  extra domains can be added to the Service Discovery item list.
\end{description}

Examples:
\begin{itemize}
\item To serve a link to the Jabber User Directory on \jid{jabber.org}:
  \begin{verbatim}
  {modules,
   [
    ...
    {mod_disco, [{extra_domains, ["users.jabber.org"]}]},
    ...
   ]}.
\end{verbatim}
\item To serve a link to the transports on another server:
  \begin{verbatim}
  {modules,
   [
    ...
    {mod_disco, [{extra_domains, ["icq.example.com",
                                  "msn.example.com"]}]},
    ...
   ]}.
\end{verbatim}
\item To serve a link to a few friendly servers:
  \begin{verbatim}
  {modules,
   [
    ...
    {mod_disco, [{extra_domains, ["example.org",
                                  "example.com"]}]},
    ...
   ]}.
\end{verbatim}
\end{itemize}


\subsection{\modecho{}}
\label{modecho}
\ind{modules!\modecho{}}\ind{debugging}

This module simply echoes any \Jabber{}
packet back to the sender. This mirror can be of interest for
\ejabberd{} and \Jabber{} client debugging.

Options:
\begin{description}
\hostitem{echo}
\end{description}

Examples:
\begin{itemize}
\item Mirror, mirror, on the wall, who is the most beautiful
  of them all?
  \begin{verbatim}
  {modules,
   [
    ...
    {mod_echo, [{hosts, ["mirror.example.org"]}]},
    ...
   ]}.
\end{verbatim}
\item If you still do not understand the inner workings of \modecho{},
  you can find a few more examples in section~\ref{modhostsoption}.
\end{itemize}

\subsection{\modirc{}}
\label{modirc}
\ind{modules!\modirc{}}\ind{IRC}

This module is an IRC transport that can be used to join channels on IRC
servers.

End user information:
\ind{protocols!groupchat 1.0}\ind{protocols!XEP-0045: Multi-User Chat}
\begin{itemize}
\item A \Jabber{} client with `groupchat 1.0' support or Multi-User
  Chat support (\xepref{0045}) is necessary to join IRC channels.
\item An IRC channel can be joined in nearly the same way as joining a
  \Jabber{} Multi-User Chat room. The difference is that the room name will
  be `channel\%\jid{irc.example.org}' in case \jid{irc.example.org} is
  the IRC server hosting `channel'. And of course the host should point
  to the IRC transport instead of the Multi-User Chat service.
\item You can register your nickame by sending `IDENTIFY password' to \\
  \jid{nickserver!irc.example.org@irc.jabberserver.org}.
\item Entering your password is possible by sending `LOGIN nick password' \\
  to \jid{nickserver!irc.example.org@irc.jabberserver.org}.
\item When using a popular \Jabber{} server, it can occur that no
  connection can be achieved with some IRC servers because they limit the
  number of conections from one IP.
\end{itemize}

Options:
\begin{description}
\hostitem{irc}
\titem{access} \ind{options!access}This option can be used to specify who
  may use the IRC transport (default value: \term{all}).
\titem{default\_encoding} \ind{options!defaultencoding}Set the default IRC encoding (default value: \term{"koi8-r"}).
\end{description}

Examples:
\begin{itemize}
\item In the first example, the IRC transport is available on (all) your
  virtual host(s) with the prefix `\jid{irc.}'. Furthermore, anyone is
  able to use the transport. The default encoding is set to "iso8859-15".
  \begin{verbatim}
  {modules,
   [
    ...
    {mod_irc, [{access, all}, {default_encoding, "iso8859-15"}]},
    ...
   ]}.
\end{verbatim}
%TODO: bug in current svn!: irc-transport.example.com will *not* show up in the
%      service discovery items; instead you will see irc.example.com
\item In next example the IRC transport is available on the two virtual hosts
  \jid{example.net} and \jid{example.com} with different prefixes on each host.
  Moreover, the transport is only accessible by paying customers registered on
  our domains and on other servers.
  \begin{verbatim}
  {acl, paying_customers, {user, "customer1", "example.net"}}.
  {acl, paying_customers, {user, "customer2", "example.com"}}.
  {acl, paying_customers, {user, "customer3", "example.org"}}.
  ...
  {access, paying_customers, [{allow, paying_customers},
                              {deny, all}]}.
  ...
  {modules,
   [
    ...
    {mod_irc, [{access, paying_customers},
               {hosts, ["irc.example.net", "irc-transport.example.com"]}]},
    ...
   ]}.
\end{verbatim}
\end{itemize}

\subsection{\modlast{}}
\label{modlast}
\ind{modules!\modlast{}}\ind{protocols!XEP-0012: Last Activity}

This module adds support for Last Activity (\xepref{0012}). It can be used to
discover when a disconnected user last accessed the server, to know when a
connected user was last active on the server, or to query the uptime of the
\ejabberd{} server.

Options:
\begin{description}
\iqdiscitem{Last activity (\ns{jabber:iq:last})}
\end{description}

\subsection{\modmuc{}}
\label{modmuc}
\ind{modules!\modmuc{}}\ind{protocols!XEP-0045: Multi-User Chat}\ind{conferencing}

With this module enabled, your server will support Multi-User Chat
(\xepref{0045}). End users will be able to join text conferences. Notice
that this module is not (yet) clusterable.


Some of the features of Multi-User Chat:
\begin{itemize}
\item Sending private messages to room participants.
\item Inviting users.
\item Setting a conference topic.
\item Creating password protected rooms.
\item Kicking and banning participants.
\end{itemize}

Options:
\begin{description}
\hostitem{conference}
\titem{access} \ind{options!access}You can specify who is allowed to use
  the Multi-User Chat service (by default, everyone is allowed to use it).
\titem{access\_create} \ind{options!access\_create}To configure who is
  allowed to create new rooms at the Multi-User Chat service, this option
  can be used (by default, everybody is allowed to create rooms).
\titem{access\_persistent} \ind{options!access\_persistent}To configure who is
  allowed to modify the 'persistent' chatroom option
  (by default, everybody is allowed to modify that option).
\titem{access\_admin} \ind{options!access\_admin}This option specifies
  who is allowed to administrate the Multi-User Chat service (the default
  value is \term{none}, which means that only the room creator can
  administer his room). By sending a message to the service JID,
  administrators can send service messages that will be displayed in every
  active room.
\titem{history\_size} \ind{options!history\_size}A small history of the
  current discussion is sent to users when they enter the room. With this option
  you can define the number of history messages to keep and send to users
  joining the room. The value is an integer. Setting the value to \term{0}
  disables the history feature and, as a result, nothing is kept in memory. The
  default value is \term{20}. This value is global and thus affects all rooms on
  the server.
\titem{min\_message\_interval} \ind{options!min\_message\_interval}
This option defines the minimum interval between two messages send by
a user in seconds. This option is global and valid for all chat
rooms. A decimal value can be used. When this option is not defined,
message rate is not limited. This feature can be used to protect a MUC
service from users abuses and limit number of messages that will be
broadcasted by the service. A good value for this minimum message
interval is 0.4 second. If a user tries to send messages faster, an
error is send back explaining that the message have been discarded and
describing the reason why the message is not acceptable.
\titem{min\_presence\_interval} \ind{options!min\_presence\_interval}
This option defines the minimum of time between presence changes
coming from a given user in seconds. This option is global and valid
for all chat rooms. A decimal value can be used. When this option is
not defined, no restriction is applied. This option can be used to
protect a MUC service for users abuses, as fastly changing a user
presence will result in possible large presence packet broadcast. If a
user tries to change its presence more often than the specified
interval, the presence is cached by ejabberd and only the last
presence is broadcasted to all users in the room after expiration of
the interval delay. Intermediate presence packets are silently
discarded. A good value for this option is 4 seconds.
\titem{default\_room\_opts} \ind{options!default\_room\_opts}This option allow
  to define the desired default room options.
  Obviously, the room creator can modify the room options at any time.
  The available room options are:
  \option{allow\_change\_subj}, \option{allow\_private\_messages},
  \option{allow\_query\_users}, \option{allow\_user\_invites},
  \option{anonymous}, \option{logging}, \option{members\_by\_default},
  \option{members\_only}, \option{moderated}, \option{password},
  \option{password\_protected}, \option{persistent},
  \option{public}, \option{public\_list}, \option{title}.
  All of them can be set to \option{true} or \option{false},
  except \option{password} and \option{title} which are strings.
\end{description}

Examples:
\begin{itemize}
\item In the first example everyone is allowed to use the Multi-User Chat
  service. Everyone will also be able to create new rooms but only the user
  \jid{admin@example.org} is allowed to administrate any room. In this
  example he is also a global administrator. When \jid{admin@example.org}
  sends a message such as `Tomorrow, the \Jabber{} server will be moved
  to new hardware. This will involve service breakdowns around 23:00 UMT.
  We apologise for this inconvenience.' to \jid{conference.example.org},
  it will be displayed in all active rooms. In this example the history
  feature is disabled.
  \begin{verbatim}
  {acl, admins, {user, "admin", "example.org"}}.
  ...
  {access, muc_admins, [{allow, admins}]}.
  ...
  {modules,
   [
    ...
    {mod_muc, [{access, all},
               {access_create, all},
               {access_admin, muc_admins},
               {history_size, 0}]},
    ...
   ]}.
\end{verbatim}
\item In the second example the Multi-User Chat service is only accessible by
  paying customers registered on our domains and on other servers. Of course
  the administrator is also allowed to access rooms. In addition, he is the
  only authority able to create and administer rooms. When
  \jid{admin@example.org} sends a message such as `Tomorrow, the \Jabber{}
  server will be moved to new hardware. This will involve service breakdowns
  around 23:00 UMT. We apologise for this inconvenience.' to
  \jid{conference.example.org}, it will be displayed in all active rooms. No
  \term{history\_size} option is used, this means that the feature is enabled
  and the default value of 20 history messages will be send to the users.
  \begin{verbatim}
  {acl, paying_customers, {user, "customer1", "example.net"}}.
  {acl, paying_customers, {user, "customer2", "example.com"}}.
  {acl, paying_customers, {user, "customer3", "example.org"}}.
  {acl, admins, {user, "admin", "example.org"}}.
  ...
  {access, muc_admins, [{allow, admins},
                        {deny, all}]}.
  {access, muc_access, [{allow, paying_customers},
                        {allow, admins},
                        {deny, all}]}.
  ...
  {modules,
   [
    ...
    {mod_muc, [{access, muc_access},
               {access_create, muc_admins},
               {access_admin, muc_admins}]},
    ...
   ]}.
\end{verbatim}

\item In the following example, MUC anti abuse options are used. A
user cannot send more than one message every 0.4 seconds and cannot
change its presence more than once every 4 seconds. No ACLs are
defined, but some user restriction could be added as well:

  \begin{verbatim}
  ...
  {modules,
   [
    ...
    {mod_muc, [{min_message_interval, 0.4},
               {min_presence_interval, 4}]},
    ...
   ]}.
\end{verbatim}

\item This example shows how to use \option{default\_room\_opts} to make sure
  newly created chatrooms have by default those options.
  \begin{verbatim}
  {modules,
   [
    ...
    {mod_muc, [{access, muc_access},
               {access_create, muc_admins},
               {default_room_options, [
                 {allow_change_subj, false},
                 {allow_query_users, true},
                 {allow_private_messages, true},
                 {members_by_default, false},
                 {title, "New chatroom"},
                 {anonymous, false}
               ]},
               {access_admin, muc_admins}]},
    ...
   ]}.
\end{verbatim}
\end{itemize}

The Multi-Users Chat module now supports clustering and load
balancing. One module can be started per cluster node. Rooms are
distributed at creation time on all available MUC module
instances. The multi-user chat module is clustered but the room
themselves are not clustered nor fault-tolerant: If the node managing a
set of rooms goes down, the rooms disappear and they will be recreated
on an available node on first connection attempt.

\subsection{\modmuclog{}}
\label{modmuclog}
\ind{modules!\modmuclog{}}

This module enables optional logging of Multi-User Chat (MUC) conversations to
HTML. Once you enable this module, users can join a chatroom using a MUC capable
Jabber client, and if they have enough privileges, they can request the
configuration form in which they can set the option to enable chatroom logging.

Features:
\begin{itemize}
\item Chatroom details are added on top of each page: room title, JID,
  author, subject and configuration.
\item \ind{protocols!RFC 4622: Internationalized Resource Identifiers (IRIs) and Uniform Resource Identifiers (URIs) for the Extensible Messaging and Presence Protocol (XMPP)}
  Room title and JID are links to join the chatroom (using
  \footahref{http://www.ietf.org/rfc/rfc4622.txt}{XMPP URIs}).
\item Subject and chatroom configuration changes are tracked and displayed.
\item Joins, leaves, nick changes, kicks, bans and `/me' are tracked and
  displayed, including the reason if available.
\item Generated HTML files are XHTML 1.0 Transitional and CSS compliant.
\item Timestamps are self-referencing links.
\item Links on top for quicker navigation: Previous day, Next day, Up.
\item CSS is used for style definition, and a custom CSS file can be used.
\item URLs on messages and subjects are converted to hyperlinks.
\item Timezone used on timestamps is shown on the log files.
\item A custom link can be added on top of each page.
\end{itemize}

Options:
\begin{description}
\titem{access\_log}\ind{options!access\_log}
  This option restricts which users are allowed to enable or disable chatroom
  logging. The default value is \term{muc\_admin}. Note for this default setting
  you need to have an access rule for \term{muc\_admin} in order to take effect.
\titem{cssfile}\ind{options!cssfile}
  With this option you can set whether the HTML files should have a custom CSS
  file or if they need to use the embedded CSS file. Allowed values are
  \term{false} and an URL to a CSS file. With the first value, HTML files will
  include the embedded CSS code. With the latter, you can specify the URL of the
  custom CSS file (for example: `http://example.com/my.css'). The default value
  is \term{false}.
\titem{dirtype}\ind{options!dirtype}
  The type of the created directories can be specified with this option. Allowed
  values are \term{subdirs} and \term{plain}. With the first value,
  subdirectories are created for each year and month. With the latter, the
  names of the log files contain the full date, and there are no subdirectories.
  The default value is \term{subdirs}.
\titem{outdir}\ind{options!outdir}
  This option sets the full path to the directory in which the HTML files should
  be stored. Make sure the \ejabberd{} daemon user has write access on that
  directory. The default value is \term{"www/muc"}.
\titem{timezone}\ind{options!timezone}
  The time zone for the logs is configurable with this option. Allowed values
  are \term{local} and \term{universal}. With the first value, the local time,
  as reported to Erlang by the operating system, will be used. With the latter,
  GMT/UTC time will be used. The default value is \term{local}.
\titem{spam\_prevention}\ind{options!spam\_prevention}
  To prevent spam, the \term{spam\_prevention} option adds a special attribute
  to links that prevent their indexation by search engines. The default value
  is \term{true}, which mean that nofollow attributes will be added to user
  submitted links. 
\titem{top\_link}\ind{options!top\_link}
  With this option you can customize the link on the top right corner of each
  log file. The syntax of this option is \term{\{"URL", "Text"\}}. The default
  value is \term{\{"/", "Home"\}}.
\end{description}

Examples:
\begin{itemize}
\item In the first example any chatroom owner can enable logging, and a
  custom CSS file will be used (http://example.com/my.css). Further, the names
  of the log files will contain the full date, and there will be no
  subdirectories. The log files will be stored in /var/www/muclogs, and the
  time zone will be GMT/UTC. Finally, the top link will be
  \verb|<a href="http://www.jabber.ru">Jabber.ru</a>|.
  \begin{verbatim}
  {access, muc, [{allow, all}]}.
  ...
  {modules,
   [
    ...
    {mod_muc_log, [
               {access_log, muc},
               {cssfile, "http://example.com/my.css"},
               {dirtype, plain},
               {outdir, "/var/www/muclogs"},
               {timezone, universal},
               {spam_prevention, true},
               {top_link, {"http://www.jabber.ru", "Jabber.ru"}}
    ]},
    ...
   ]}.
\end{verbatim}
  \item In the second example only \jid{admin1@example.org} and
  \jid{admin2@example.net} can enable logging, and the embedded CSS file will be
  used. Further, the names of the log files will only contain the day (number),
  and there will be subdirectories for each year and month. The log files will
  be stored in /var/www/muclogs, and the local time will be used. Finally, the
  top link will be the default \verb|<a href="/">Home</a>|.
  \begin{verbatim}
  {acl, admins, {user, "admin1", "example.org"}}.
  {acl, admins, {user, "admin2", "example.net"}}.
  ...
  {access, muc_log, [{allow, admins},
                     {deny, all}]}.
  ...
  {modules,
   [
    ...
    {mod_muc_log, [
               {access_log, muc_log},
               {cssfile, false},
               {dirtype, subdirs},
               {outdir, "/var/www/muclogs"},
               {timezone, local}
    ]},
    ...
   ]}.
\end{verbatim}
\end{itemize}

\subsection{\modoffline{}}
\label{modoffline}
\ind{modules!\modoffline{}}

This module implements offline message storage. This means that all messages
sent to an offline user will be stored on the server until that user comes
online again. Thus it is very similar to how email works. Note that
\term{ejabberdctl}\ind{ejabberdctl} has a command to delete expired messages
(see section~\ref{ejabberdctl}).

\begin{description}
  \titem{user\_max\_messages}\ind{options!user\_max\_messages}This option
  is use to set a max number of offline messages per user (quota). Its
  value can be either \term{infinity} or a strictly positive
  integer. The default value is \term{infinity}.
\end{description}


\subsection{\modprivacy{}}
\label{modprivacy}
\ind{modules!\modprivacy{}}\ind{Blocking Communication}\ind{Privacy Rules}\ind{protocols!RFC 3921: XMPP IM}

This module implements Blocking Communication (also known as Privacy Rules)
as defined in section 10 from XMPP IM. If end users have support for it in
their \Jabber{} client, they will be able to:
\begin{quote}
\begin{itemize}
\item Retrieving one's privacy lists.
\item Adding, removing, and editing one's privacy lists.
\item Setting, changing, or declining active lists.
\item Setting, changing, or declining the default list (i.e., the list that
  is active by default).
\item Allowing or blocking messages based on JID, group, or subscription type
  (or globally).
\item Allowing or blocking inbound presence notifications based on JID, group,
  or subscription type (or globally).
\item Allowing or blocking outbound presence notifications based on JID, group,
  or subscription type (or globally).
\item Allowing or blocking IQ stanzas based on JID, group, or subscription type
  (or globally).
\item Allowing or blocking all communications based on JID, group, or
  subscription type (or globally).
\end{itemize}
(from \ahrefurl{http://www.xmpp.org/specs/rfc3921.html\#privacy})
\end{quote}

Options:
\begin{description}
\iqdiscitem{Blocking Communication (\ns{jabber:iq:privacy})}
\end{description}

\subsection{\modprivate{}}
\label{modprivate}
\ind{modules!\modprivate{}}\ind{protocols!XEP-0049: Private XML Storage}\ind{protocols!XEP-0048: Bookmark Storage}

This module adds support for Private XML Storage (\xepref{0049}):
\begin{quote}
Using this method, Jabber entities can store private data on the server and
retrieve it whenever necessary. The data stored might be anything, as long as
it is valid XML. One typical usage for this namespace is the server-side storage
of client-specific preferences; another is Bookmark Storage (\xepref{0048}).
\end{quote}

Options:
\begin{description}
\iqdiscitem{Private XML Storage (\ns{jabber:iq:private})}
\end{description}

\subsection{\modproxy{}}
\label{modproxy}
\ind{modules!\modversion{}}\ind{protocols!XEP-0065: SOCKS5 Bytestreams}

This module implements SOCKS5 Bytestreams (\xepref{0065}).
It allows \ejabberd{} to act as a file transfer proxy between two
XMPP clients.

Options:
\begin{description}
\titem{host}\ind{options!host}This option defines the hostname of the service.
If this option is not set, the prefix `\jid{proxy.}' is added to \ejabberd{}
hostname.
\titem{name}\ind{options!name}Defines Service Discovery name of the service.
Default is \term{"SOCKS5 Bytestreams"}.
\titem{ip}\ind{options!ip}This option specifies which network interface
to listen for. Default is an IP address of the service's DNS name, or,
if fails, \verb|{127,0,0,1}|.
\titem{port}\ind{options!port}This option defines port to listen for
incoming connections. Default is~7777.
\titem{auth\_type}\ind{options!auth\_type}SOCKS5 authentication type.
Possible values are \term{anonymous} and \term{plain}. Default is
\term{anonymous}.
\titem{access}\ind{options!access}Defines ACL for file transfer initiators.
Default is \term{all}.
\titem{max\_connections}\ind{options!max\_connections}Maximum number of
active connections per file transfer initiator. No limit by default.
\titem{shaper}\ind{options!shaper}This option defines shaper for
the file transfer peers. Shaper with the maximum bandwidth will be selected.
Default is \term{none}.
\end{description}

Examples:
\begin{itemize}
\item The simpliest configuration of the module:
  \begin{verbatim}
  {modules,
   [
    ...
    {mod_proxy65, []},
    ...
  ]}.
\end{verbatim}
\item More complicated configuration.
  \begin{verbatim}
  {acl, proxy_users, {server, "example.org"}}.
  {access, proxy65_access, [{allow, proxy_users}, {deny, all}]}.
  ...
  {acl, admin, {user, "admin", "example.org"}}.
  {shaper, normal, {maxrate, 10240}}. %% 10 Kbytes/sec
  {access, proxy65_shaper, [{none, admin}, {normal, all}]}.
  ...
  {modules,
   [
    ...
    {mod_proxy65, [{host, "proxy1.example.org"},
                   {name, "File Transfer Proxy"},
                   {ip, {200,150,100,1}},
                   {port, 7778},
                   {max_connections, 5},
                   {access, proxy65_access},
                   {shaper, proxy65_shaper}]},
    ...
  ]}.
\end{verbatim}
\end{itemize}

\subsection{\modpubsub{}}
\label{modpubsub}
\ind{modules!\modpubsub{}}\ind{protocols!XEP-0060: Publish-Subscribe}

This module offers a Publish-Subscribe Service (\xepref{0060}).
Publish-Subscribe can be used to develop (examples are taken from the XEP):
\begin{quote}
\begin{itemize}
\item news feeds and content syndacation,
\item avatar management,
\item shared bookmarks,
\item auction and trading systems,
\item online catalogs,
\item workflow systems,
\item network management systems,
\item NNTP gateways,
\item vCard/profile management,
\item and weblogs.
\end{itemize}
\end{quote} 

\ind{J-EAI}\ind{EAI}\ind{ESB}\ind{Enterprise Application Integration}\ind{Enterprise Service Bus}
Another example is \footahref{http://www.process-one.net/en/projects/j-eai/}{J-EAI}.
This is an XMPP-based Enterprise Application Integration (EAI) platform (also
known as ESB, the Enterprise Service Bus). The J-EAI project builts upon
\ejabberd{}'s codebase and has contributed several features to \modpubsub{}.

Options:
\begin{description}
\hostitem{pubsub}
\titem{served\_hosts} \ind{options!served\_hosts}To specify which hosts needs to
  be served, you can use this option. If absent, only the main \ejabberd{}
  host is served. % Not a straigtforward description! This needs to be improved!
\titem{access\_createnode} \ind{options!access\_createnode}
  This option restricts which users are allowed to create pubsub nodes using
  ACL and ACCESS. The default value is \term{pubsub\_createnode}. % Not clear enough + do not use abbreviations.
\end{description}

Example:
\begin{verbatim}
  {modules,
   [
    ...
    {mod_pubsub, [{served_hosts, ["example.com",
                                  "example.org"]},
                  {access_createnode, pubsub_createnode}]}
    ...
   ]}.
\end{verbatim}

\subsection{\modregister{}}
\label{modregister}
\ind{modules!\modregister{}}\ind{protocols!XEP-0077: In-Band Registration}\ind{public registration}

This module adds support for In-Band Registration (\xepref{0077}). This protocol
enables end users to use a \Jabber{} client to:
\begin{itemize}
\item Register a new account on the server.
\item Change the password from an existing account on the server.
\item Delete an existing account on the server.
\end{itemize}


Options:
\begin{description}
\titem{access} \ind{options!access}This option can be configured to specify
  rules to restrict registration. If a rule returns `deny' on the requested
  user name, registration for that user name is dennied. (there are no
  restrictions by default).
\iqdiscitem{In-Band Registration (\ns{jabber:iq:register})}
\end{description}

Examples:
\begin{itemize}
\item Next example prohibits the registration of too short account names:
\begin{verbatim}
  {acl, shortname, {user_glob, "?"}}.
  {acl, shortname, {user_glob, "??"}}.
  % The same using regexp:
  %{acl, shortname, {user_regexp, "^..?$"}}.
  ...
  {access, register, [{deny, shortname},
                      {allow, all}]}.
  ...
  {modules,
   [
    ...
    {mod_register, [{access, register}]},
    ...
   ]}.
\end{verbatim}
\item The in-band registration of new accounts can be prohibited by changing the
  \option{access} option. If you really want to disable all In-Band Registration
  functionality, that is changing passwords in-band and deleting accounts
  in-band, you have to remove \modregister{} from the modules list. In this
  example all In-Band Registration functionality is disabled:
  \begin{verbatim}
  {access, register, [{deny, all}]}.

  {modules,
   [
    ...
%    {mod_register, [{access, register}]},
    ...
   ]}.
\end{verbatim}
\end{itemize}

\subsection{\modroster{}}
\label{modroster}
\ind{modules!\modroster{}}\ind{roster management}\ind{protocols!RFC 3921: XMPP IM}

This module implements roster management as defined in \footahref{http://www.xmpp.org/specs/rfc3921.html\#roster}{RFC 3921: XMPP IM}.

Options:
\begin{description}
\iqdiscitem{Roster Management (\ns{jabber:iq:roster})}
\end{description}

\subsection{\modservicelog{}}
\label{modservicelog}
\ind{modules!\modservicelog{}}\ind{message auditing}\ind{Bandersnatch}

This module adds support for logging end user packets via a \Jabber{} message
auditing service such as
\footahref{http://www.funkypenguin.co.za/bandersnatch/}{Bandersnatch}. All user
packets are encapsulated in a \verb|<route/>| element and sent to the specified
service(s).

Options:
\begin{description}
\titem{loggers} \ind{options!loggers}With this option a (list of) service(s)
  that will receive the packets can be specified.
\end{description}

Examples:
\begin{itemize}
\item To log all end user packets to the Bandersnatch service running on
  \jid{bandersnatch.example.com}:
  \begin{verbatim}
  {modules,
   [
    ...
    {mod_service_log, [{loggers, ["bandersnatch.example.com"]}]},
    ...
   ]}.
\end{verbatim}
\item To log all end user packets to the Bandersnatch service running on
  \jid{bandersnatch.example.com} and the backup service on 
  \jid{bandersnatch.example.org}:
  \begin{verbatim}
  {modules,
   [
    ...
    {mod_service_log, [{loggers, ["bandersnatch.example.com",
                                  "bandersnatch.example.org"]}]},
    ...
   ]}.
\end{verbatim}
\end{itemize}

\subsection{\modsharedroster{}}
\label{modsharedroster}
\ind{modules!\modsharedroster{}}\ind{shared roster groups}

This module enables you to create shared roster groups. This means that you can
create groups of people that can see members from (other) groups in their
rosters. The big advantages of this feature are that end users do not need to
manually add all users to their rosters, and that they cannot permanently delete
users from the shared roster groups.

Shared roster groups can be edited \emph{only} via the web interface. Each group
has a unique identification and the following parameters:
\begin{description}
\item[Name] The name of the group, which will be displayed in the roster.
\item[Description] The description of the group. This parameter does not affect
  anything.
\item[Members] A list of full JIDs of group members, entered one per line in
  the web interface.
\item[Displayed groups] A list of groups that will be in the rosters of this
  group's members.
\end{description}

Examples:
\begin{itemize}
\item Take the case of a computer club that wants all its members seeing each
  other in their rosters. To achieve this, they need to create a shared roster
  group similar to next table:
\begin{table}[H]
  \centering
  \begin{tabular}{|l|l|}
    \hline Identification& Group `\texttt{club\_members}'\\
    \hline Name& Club Members\\
    \hline Description& Members from the computer club\\
    \hline Members&
    {\begin{tabular}{l}
        \jid{member1@example.org}\\
        \jid{member2@example.org}\\
        \jid{member3@example.org}
      \end{tabular}
    }\\
    \hline Displayed groups& \texttt{club\_members}\\
    \hline
  \end{tabular}
\end{table}
\item In another case we have a company which has three divisions: Management,
  Marketing and Sales. All group members should see all other members in their
  rosters. Additonally, all managers should have all marketing and sales people
  in their roster. Simultaneously, all marketeers and the whole sales team
  should see all managers. This scenario can be achieved by creating shared
  roster groups as shown in the following table:
\begin{table}[H]
  \centering
  \begin{tabular}{|l|l|l|l|}
    \hline Identification&
    Group `\texttt{management}'&
    Group `\texttt{marketing}'&
    Group `\texttt{sales}'\\
    \hline Name& Management& Marketing& Sales\\
    \hline Description& \\
    Members&
    {\begin{tabular}{l}
        \jid{manager1@example.org}\\
        \jid{manager2@example.org}\\
        \jid{manager3@example.org}\\
        \jid{manager4@example.org}
      \end{tabular}
    }&
    {\begin{tabular}{l}
        \jid{marketeer1@example.org}\\
        \jid{marketeer2@example.org}\\
        \jid{marketeer3@example.org}\\
        \jid{marketeer4@example.org}
      \end{tabular}
    }&
    {\begin{tabular}{l}
        \jid{saleswoman1@example.org}\\
        \jid{salesman1@example.org}\\
        \jid{saleswoman2@example.org}\\
        \jid{salesman2@example.org}
      \end{tabular}
    }\\
    \hline Displayed groups&
    {\begin{tabular}{l}
        \texttt{management}\\
        \texttt{marketing}\\
        \texttt{sales}
      \end{tabular}
    }&
    {\begin{tabular}{l}
        \texttt{management}\\
        \texttt{marketing}
      \end{tabular}
    }&
    {\begin{tabular}{l}
        \texttt{management}\\
        \texttt{sales}
      \end{tabular}
    }\\
    \hline
  \end{tabular}
\end{table}
\end{itemize}

\subsection{\modstats{}}
\label{modstats}
\ind{modules!\modstats{}}\ind{protocols!XEP-0039: Statistics Gathering}\ind{statistics}

This module adds support for Statistics Gathering (\xepref{0039}). This protocol
allows you to retrieve next statistics from your \ejabberd{} deployment:
\begin{itemize}
\item Total number of registered users on the current virtual host (users/total).
\item Total number of registered users on all virtual hosts (users/all-hosts/total).
\item Total number of online users on the current virtual host (users/online).
\item Total number of online users on all virtual hosts (users/all-hosts/online).
\end{itemize}

Options:
\begin{description}
\iqdiscitem{Statistics Gathering (\ns{http://jabber.org/protocol/stats})}
\end{description}

As there are only a small amount of clients (for \ind{Tkabber}example
\footahref{http://tkabber.jabber.ru/}{Tkabber}) and software libraries with
support for this XEP, a few examples are given of the XML you need to send
in order to get the statistics. Here they are:
\begin{itemize}
\item You can request the number of online users on the current virtual host
  (\jid{example.org}) by sending:
  \begin{verbatim}
<iq to='example.org' type='get'>
  <query xmlns='http://jabber.org/protocol/stats'>
    <stat name='users/online'/>
  </query>
</iq>
\end{verbatim}
\item You can request the total number of registered users on all virtual hosts
  by sending:
  \begin{verbatim}
<iq to='example.org' type='get'>
  <query xmlns='http://jabber.org/protocol/stats'>
    <stat name='users/all-hosts/total'/>
  </query>
</iq>
\end{verbatim}
\end{itemize}

\subsection{\modtime{}}
\label{modtime}
\ind{modules!\modtime{}}\ind{protocols!XEP-0090: Entity Time}

This module features support for Entity Time (\xepref{0090}). By using this XEP,
you are able to discover the time at another entity's location.

Options:
\begin{description}
\iqdiscitem{Entity Time (\ns{jabber:iq:time})}
\end{description}

\subsection{\modvcard{}}
\label{modvcard}
\ind{modules!\modvcard{}}\ind{JUD}\ind{Jabber User Directory}\ind{vCard}\ind{protocols!XEP-0054: vcard-temp}

This module allows end users to store and retrieve their vCard, and to retrieve
other users vCards, as defined in vcard-temp (\xepref{0054}). The module also
implements an uncomplicated \Jabber{} User Directory based on the vCards of
these users. Moreover, it enables the server to send its vCard when queried.

Options:
\begin{description}
\hostitem{vjud}
\iqdiscitem{\ns{vcard-temp}}
\titem{search}\ind{options!search}This option specifies whether the search
  functionality is enabled (value: \term{true}) or disabled (value:
  \term{false}). If disabled, the option \term{hosts} will be ignored and the
  \Jabber{} User Directory service will not appear in the Service Discovery item
  list. The default value is \term{true}.
\titem{matches}\ind{options!matches}With this option, the number of reported
  search results can be limited. If the option's value is set to \term{infinity},
  all search results are reported. The default value is \term{30}.
\titem{allow\_return\_all}\ind{options!allow\_return\_all}This option enables
  you to specify if search operations with empty input fields should return all
  users who added some information to their vCard. The default value is
  \term{false}.
\titem{search\_all\_hosts}\ind{options!search\_all\_hosts}If this option is set
  to \term{true}, search operations will apply to all virtual hosts. Otherwise
  only the current host will be searched. The default value is \term{true}.
\end{description}

Examples:
\begin{itemize}
\item In this first situation, search results are limited to twenty items,
  every user who added information to their vCard will be listed when people
  do an empty search, and only users from the current host will be returned:
  \begin{verbatim}
  {modules,
   [
    ...
    {mod_vcard, [{search, true},
                 {matches, 20},
                 {allow_return_all, true},
                 {search_all_hosts, false}]},
    ...
   ]}.
\end{verbatim}
\item The second situation differs in a way that search results are not limited,
  and that all virtual hosts will be searched instead of only the current one:
  \begin{verbatim}
  {modules,
   [
    ...
    {mod_vcard, [{search, true},
                 {matches, infinity},
                 {allow_return_all, true}]},
    ...
   ]}.
\end{verbatim}
\end{itemize}

\subsection{\modvcardldap{}}
\label{modvcardldap}
\ind{modules!\modvcardldap{}}\ind{JUD}\ind{Jabber User Directory}\ind{vCard}\ind{protocols!XEP-0054: vcard-temp}

%TODO: verify if the referers to the LDAP section are still correct

\ejabberd{} can map LDAP attributes to vCard fields. This behaviour is
implemented in the \modvcardldap{} module. This module does not depend on the
authentication method (see~\ref{ldapauth}). The \modvcardldap{} module has
its own optional parameters. The first group of parameters has the same
meaning as the top-level LDAP parameters to set the authentication method:
\option{ldap\_servers}, \option{ldap\_port}, \option{ldap\_rootdn},
\option{ldap\_password}, \option{ldap\_base}, \option{ldap\_uids}, and
\option{ldap\_filter}. See section~\ref{ldapauth} for detailed information
about these options. If one of these options is not set, \ejabberd{} will look
for the top-level option with the same name. The second group of parameters
consists of the following \modvcardldap{}-specific options:

\begin{description}
\hostitem{vjud}
\iqdiscitem{\ns{vcard-temp}}
\titem{search}\ind{options!search}This option specifies whether the search
  functionality is enabled (value: \term{true}) or disabled (value:
  \term{false}). If disabled, the option \term{hosts} will be ignored and the
  \Jabber{} User Directory service will not appear in the Service Discovery item
  list. The default value is \term{true}.
\titem{ldap\_vcard\_map}\ind{options!ldap\_vcard\_map}With this option you can
  set the table that maps LDAP attributes to vCard fields. The format is:
  \term{[{Name\_of\_vCard\_field, Pattern, List\_of\_LDAP\_attributes}, ...]}.\ind{protocols!RFC 2426: vCard MIME Directory Profile}
  \term{Name\_of\_vcard\_field} is the type name of the vCard as defined in
  \footahref{http://www.ietf.org/rfc/rfc2426.txt}{RFC 2426}. \term{Pattern} is a
  string which contains pattern variables \term{"\%u"}, \term{"\%d"} or
  \term{"\%s"}. \term{List\_of\_LDAP\_attributes} is the list containing LDAP
  attributes. The pattern variables \term{"\%s"} will be sequentially replaced
  with the values of LDAP attributes from \term{List\_of\_LDAP\_attributes},
  \term{"\%u"} will be replaced with the user part of a JID, and \term{"\%d"}
  will be replaced with the domain part of a JID. The default is:
  \begin{verbatim}
  [{"NICKNAME", "%u", []},
   {"FN", "%s", ["displayName"]},
   {"FAMILY", "%s", ["sn"]},
   {"GIVEN", "%s", ["givenName"]},
   {"MIDDLE", "%s", ["initials"]},
   {"ORGNAME", "%s", ["o"]},
   {"ORGUNIT", "%s", ["ou"]},
   {"CTRY", "%s", ["c"]},
   {"LOCALITY", "%s", ["l"]},
   {"STREET", "%s", ["street"]},
   {"REGION", "%s", ["st"]},
   {"PCODE", "%s", ["postalCode"]},
   {"TITLE", "%s", ["title"]},
   {"URL", "%s", ["labeleduri"]},
   {"DESC", "%s", ["description"]},
   {"TEL", "%s", ["telephoneNumber"]},
   {"EMAIL", "%s", ["mail"]},
   {"BDAY", "%s", ["birthDay"]},
   {"ROLE", "%s", ["employeeType"]},
   {"PHOTO", "%s", ["jpegPhoto"]}]
\end{verbatim}
\titem{ldap\_search\_fields}\ind{options!ldap\_search\_fields}This option
  defines the search form and the LDAP attributes to search within. The format
  is: \term{[{Name, Attribute}, ...]}. \term{Name} is the name of a search form
  field which will be automatically translated by using the translation
  files (see \term{msgs/*.msg} for available words). \term{Attribute} is the
  LDAP attribute or the pattern \term{"\%u"}. The default is:
  \begin{verbatim}
  [{"User", "%u"},
   {"Full Name", "displayName"},
   {"Given Name", "givenName"},
   {"Middle Name", "initials"},
   {"Family Name", "sn"},
   {"Nickname", "%u"},
   {"Birthday", "birthDay"},
   {"Country", "c"},
   {"City", "l"},
   {"Email", "mail"},
   {"Organization Name", "o"},
   {"Organization Unit", "ou"}]
\end{verbatim}
\titem{ldap\_search\_reported}\ind{options!ldap\_search\_reported}This option
  defines which search fields should be reported. The format is:
  \term{[{Name, vCard\_Name}, ...]}. \term{Name} is the name of a search form
  field which will be automatically translated by using the translation
  files (see \term{msgs/*.msg} for available words). \term{vCard\_Name} is the
  vCard field name defined in the \option{ldap\_vcard\_map} option. The default
  is:
\begin{verbatim}
  [{"Full Name", "FN"},
   {"Given Name", "GIVEN"},
   {"Middle Name", "MIDDLE"},
   {"Family Name", "FAMILY"},
   {"Nickname", "NICKNAME"},
   {"Birthday", "BDAY"},
   {"Country", "CTRY"},
   {"City", "LOCALITY"},
   {"Email", "EMAIL"},
   {"Organization Name", "ORGNAME"},
   {"Organization Unit", "ORGUNIT"}]
\end{verbatim}
\end{description}

%TODO: this examples still should be organised better
Examples:
\begin{itemize}
\item 

Let's say \term{ldap.example.org} is the name of our LDAP server. We have
users with their passwords in \term{"ou=Users,dc=example,dc=org"} directory.
Also we have addressbook, which contains users emails and their additional
infos in \term{"ou=AddressBook,dc=example,dc=org"} directory.  Corresponding
authentication section should looks like this:

\begin{verbatim}
  %% authentication method
  {auth_method, ldap}.
  %% DNS name of our LDAP server
  {ldap_servers, ["ldap.example.org"]}.
  %% We want to authorize users from 'shadowAccount' object class only
  {ldap_filter, "(objectClass=shadowAccount)"}.
\end{verbatim}

Now we want to use users LDAP-info as their vCards. We have four attributes
defined in our LDAP schema: \term{"mail"} --- email address, \term{"givenName"}
--- first name, \term{"sn"} --- second name, \term{"birthDay"} --- birthday.
Also we want users to search each other. Let's see how we can set it up:

\begin{verbatim}
  {modules,
    ...
    {mod_vcard_ldap,
     [
      %% We use the same server and port, but want to bind anonymously because
      %% our LDAP server accepts anonymous requests to
      %% "ou=AddressBook,dc=example,dc=org" subtree.
      {ldap_rootdn, ""},
      {ldap_password, ""},
      %% define the addressbook's base
      {ldap_base, "ou=AddressBook,dc=example,dc=org"},
      %% uidattr: user's part of JID is located in the "mail" attribute
      %% uidattr_format: common format for our emails
      {ldap_uids, [{"mail","%u@mail.example.org"}]},
      %% We have to define empty filter here, because entries in addressbook does not
      %% belong to shadowAccount object class
      {ldap_filter, ""},
      %% Now we want to define vCard pattern
      {ldap_vcard_map,
       [{"NICKNAME", "%u", []}, % just use user's part of JID as his nickname
        {"GIVEN", "%s", ["givenName"]},
        {"FAMILY", "%s", ["sn"]},
        {"FN", "%s, %s", ["sn", "givenName"]}, % example: "Smith, John"
        {"EMAIL", "%s", ["mail"]},
        {"BDAY", "%s", ["birthDay"]}]},
      %% Search form
      {ldap_search_fields,
       [{"User", "%u"},
        {"Name", "givenName"},
        {"Family Name", "sn"},
        {"Email", "mail"},
        {"Birthday", "birthDay"}]},
      %% vCard fields to be reported
      %% Note that JID is always returned with search results
      {ldap_search_reported,
       [{"Full Name", "FN"},
        {"Nickname", "NICKNAME"},
        {"Birthday", "BDAY"}]}
    ]}
    ...
  }.
\end{verbatim}

Note that \modvcardldap{} module checks an existence of the user before
searching his info in LDAP.

\item \term{ldap\_vcard\_map} example:
\begin{verbatim}
  {ldap_vcard_map,
   [{"NICKNAME", "%u", []},
    {"FN", "%s", ["displayName"]},
    {"CTRY", "Russia", []},
    {"EMAIL", "%u@%d", []},
    {"DESC", "%s\n%s", ["title", "description"]}
   ]},
\end{verbatim}
\item \term{ldap\_search\_fields} example:
\begin{verbatim}
  {ldap_search_fields,
   [{"User", "uid"},
    {"Full Name", "displayName"},
    {"Email", "mail"}
   ]},
\end{verbatim}
\item \term{ldap\_search\_reported} example:
\begin{verbatim}
  {ldap_search_reported,
   [{"Full Name", "FN"},
    {"Email", "EMAIL"},
    {"Birthday", "BDAY"},
    {"Nickname", "NICKNAME"}
   ]},
\end{verbatim}
\end{itemize}

\subsection{\modversion{}}
\label{modversion}
\ind{modules!\modversion{}}\ind{protocols!XEP-0092: Software Version}

This module implements Software Version (\xepref{0092}). Consequently, it
answers \ejabberd{}'s version when queried.

Options:
\begin{description}
\titem{show\_os}\ind{options!showos}Should the operating system be revealed or not.
  The default value is \term{true}.
\iqdiscitem{Software Version (\ns{jabber:iq:version})}
\end{description}

\chapter{Managing an ejabberd server}
\section{Online Configuration and Monitoring}
\label{onlineconfig}

\subsection{Web Interface}
\label{webinterface}
\ind{web interface}

To perform online configuration of \ejabberd{} you need to enable the
\term{ejabberd\_http} listener with the option \term{web\_admin} (see
section~\ref{listened}). Then you can open 
\verb|http://server:port/admin/| in your favourite web browser. You
will be asked to enter the username (the \emph{full} \Jabber{} ID) and password
of an \ejabberd{} user with administrator rights. After authentication
you will see a page similar to figure~\ref{fig:webadmmain}.

\begin{figure}[htbp]
  \centering
  \insimg{webadmmain.png}
  \caption{Top page from the web interface}
  \label{fig:webadmmain}
\end{figure}
Here you can edit access restrictions, manage users, create backups,
manage the database, enable/disable ports listened for, view server
statistics,\ldots

Examples:
\begin{itemize}
\item You can serve the web interface on the same port as the
  \ind{protocols!XEP-0025: HTTP Polling}HTTP Polling interface. In this example
  you should point your web browser to \verb|http://example.org:5280/admin/| to
  administer all virtual hosts or to
  \verb|http://example.org:5280/admin/server/example.com/| to administer only
  the virtual host \jid{example.com}. Before you get access to the web interface
  you need to enter as username, the JID and password from a registered user
  that is allowed to configure \ejabberd{}. In this example you can enter as
  username `\jid{admin@example.net}' to administer all virtual hosts (first
  URL). If you log in with `\jid{admin@example.com}' on \\
  \verb|http://example.org:5280/admin/server/example.com/| you can only
  administer the virtual host \jid{example.com}.
  \begin{verbatim}
  ...
  {acl, admins, {user, "admin", "example.net"}}.
  {host_config, "example.com", [{acl, admins, {user, "admin", "example.com"}}]}.
  {access, configure, [{allow, admins}]}.
  ...
  {hosts, ["example.org"]}.
  ...
  {listen,
   [...
    {5280, ejabberd_http, [http_poll, web_admin]},
    ...
   ]
  }.
\end{verbatim}
\item For security reasons, you can serve the web interface on a secured
  connection, on a port differing from the HTTP Polling interface, and bind it
  to the internal LAN IP. The web interface will be accessible by pointing your
  web browser to \verb|https://192.168.1.1:5280/admin/|:
  \begin{verbatim}
  ...
  {hosts, ["example.org"]}.
  ...
  {listen,
   [...
    {5270, ejabberd_http,    [http_poll]},
    {5280, ejabberd_http,    [web_admin, {ip, {192, 168, 1, 1}},
                              tls, {certfile, "/usr/local/etc/server.pem"}]},
    ...
   ]
  }.
\end{verbatim}
\end{itemize}

\subsection{\term{ejabberdctl}}
\label{ejabberdctl}
%TODO: update when the ejabberdctl script is made more userfriendly

It is possible to do some administration operations using the command
line tool \term{ejabberdctl}. You can list all available options by
running \term{ejabberdctl} without arguments:
\begin{verbatim}
% ejabberdctl
Usage: ejabberdctl node command

Available commands:
  status                        get ejabberd status
  stop                          stop ejabberd
  restart                       restart ejabberd
  reopen-log                    reopen log file
  register user server password register a user
  unregister user server        unregister a user
  backup file                   store a database backup to file
  restore file                  restore a database backup from file
  install-fallback file         install a database fallback from file
  dump file                     dump a database to a text file
  load file                     restore a database from a text file
  import-file file              import user data from jabberd 1.4 spool file
  import-dir dir                import user data from jabberd 1.4 spool directory
  registered-users              list all registered users
  delete-expired-messages       delete expired offline messages from database

Example:
  ejabberdctl ejabberd@host restart
\end{verbatim}

Additional information:
\begin{description}
\titem{reopen-log } If you use a tool to rotate logs, you have to configure it
  so that this command is executed after each rotation.
\titem {backup, restore, install-fallback, dump, load} You can use these
  commands to create and restore backups. 
%%More information about backuping can
%%  be found in section~\ref{backup}.
\titem{import-file, import-dir} \ind{migration from other software}
  These options can be used to migrate from other \Jabber{}/XMPP servers. There
  exist tutorials to \footahref{http://ejabberd.jabber.ru/migrate-to-ejabberd}{migrate from other software to ejabberd}.
\titem{delete-expired-messages} This option can be used to delete old messages
  in offline storage. This might be useful when the number of offline messages
  is very high.
\end{description}

\chapter{Securing ejabberd}
\section{Firewall Settings}
\label{firewall}
\ind{firewall}\ind{ports}\ind{SASL}\ind{TLS}\ind{clustering!ports}

You need to take the following TCP ports in mind when configuring your firewall:
\begin{table}[H]
  \centering
  \begin{tabular}{|l|l|}
    \hline Port& Description\\
    \hline \hline 5222& SASL and unencrypted c2s connections.\\
    \hline 5223& Obsolete SSL c2s connections.\\
    \hline 5269& s2s connections.\\
    \hline 4369& Only for clustering (see~\ref{clustering}).\\
    \hline port range& Only for clustring (see~\ref{clustering}). This range
    is configurable (see~\ref{start}).\\
    \hline
  \end{tabular}
\end{table}

\chapter{Integrating ejabberd with other Instant Messaging servers}
\section{SRV Records}
\label{srv}
\ind{SRV Records}\ind{clustering!SRV Records}

\begin{itemize}
\item General information:
  \footahref{http://en.wikipedia.org/wiki/SRV\_record}{SRV record}
\item Practical information:
  \footahref{http://jabberd.jabberstudio.org/2/docs/section05.html\#5\_7}{Setting DNS SRV Records}
\end{itemize}

\chapter{Clustering}
\label{clustering}
\ind{clustering}

\section{How it Works}
\label{howitworks}
\ind{clustering!how it works}

A \Jabber{} domain is served by one or more \ejabberd{} nodes. These nodes can
be run on different machines that are connected via a network. They all
must have the ability to connect to port 4369 of all another nodes, and must
have the same magic cookie (see Erlang/OTP documentation, in other words the
file \term{\~{}ejabberd/.erlang.cookie} must be the same on all nodes). This is
needed because all nodes exchange information about connected users, s2s
connections, registered services, etc\ldots

Each \ejabberd{} node has the following modules:
\begin{itemize}
\item router,
\item local router,
\item session manager,
\item s2s manager.
\end{itemize}

\subsection{Router}
\label{router}
\ind{clustering!router}

This module is the main router of \Jabber{} packets on each node. It
routes them based on their destination's domains. It uses a global
routing table. The domain of the packet's destination is searched in the
routing table, and if it is found, the packet is routed to the
appropriate process. If not, it is sent to the s2s manager.

\subsection{Local Router}
\label{localrouter}
\ind{clustering!local router}

This module routes packets which have a destination domain equal to
one of this server's host names. If the destination JID has a non-empty user
part, it is routed to the session manager, otherwise it is processed depending
on its content.

\subsection{Session Manager}
\label{sessionmanager}
\ind{clustering!session manager}

This module routes packets to local users. It looks up to which user
resource a packet must be sent via a presence table. Then the packet is
either routed to the appropriate c2s process, or stored in offline
storage, or bounced back.

\subsection{s2s Manager}
\label{s2smanager}
\ind{clustering!s2s manager}

This module routes packets to other \Jabber{} servers. First, it
checks if an opened s2s connection from the domain of the packet's
source to the domain of the packet's destination exists. If that is the case,
the s2s manager routes the packet to the process
serving this connection, otherwise a new connection is opened.

\section{Clustering Setup}
\label{cluster}
\ind{clustering!setup}

Suppose you already configured \ejabberd{} on one machine named (\term{first}),
and you need to setup another one to make an \ejabberd{} cluster. Then do
following steps:

\begin{enumerate}
\item Copy \verb|~ejabberd/.erlang.cookie| file from \term{first} to
  \term{second}.

  (alt) You can also add `\verb|-cookie content_of_.erlang.cookie|'
  option to all `\shell{erl}' commands below.

\item On \term{second} run the following command as the \ejabberd{} daemon user,
  in the working directory of \ejabberd{}:

\begin{verbatim}
erl -sname ejabberd \
    -mnesia extra_db_nodes "['ejabberd@first']" \
    -s mnesia
\end{verbatim}

  This will start Mnesia serving the same database as \node{ejabberd@first}.
  You can check this by running the command `\verb|mnesia:info().|'. You
  should see a lot of remote tables and a line like the following:

\begin{verbatim}
running db nodes   = [ejabberd@first, ejabberd@second]
\end{verbatim}


\item Now run the following in the same `\shell{erl}' session:

\begin{verbatim}
mnesia:change_table_copy_type(schema, node(), disc_copies).
\end{verbatim}

  This will create local disc storage for the database.

  (alt) Change storage type of the \term{scheme} table to `RAM and disc
  copy' on the second node via the web interface.


\item Now you can add replicas of various tables to this node with
  `\verb|mnesia:add_table_copy|' or
  `\verb|mnesia:change_table_copy_type|' as above (just replace
  `\verb|schema|' with another table name and `\verb|disc_copies|'
  can be replaced with `\verb|ram_copies|' or
  `\verb|disc_only_copies|').

  Which tables to replicate is very dependant on your needs, you can get
  some hints from the command `\verb|mnesia:info().|', by looking at the
  size of tables and the default storage type for each table on 'first'.

  Replicating a table makes lookups in this table faster on this node.
  Writing, on the other hand, will be slower. And of course if machine with one
  of the replicas is down, other replicas will be used.

  Also \footahref{http://www.erlang.se/doc/doc-5.4.9/lib/mnesia-4.2.2/doc/html/Mnesia\_chap5.html\#5.3}
  {section 5.3 (Table Fragmentation) of Mnesia User's Guide} can be helpful.
  % The above URL needs update every Erlang release!

  (alt) Same as in previous item, but for other tables.


\item Run `\verb|init:stop().|' or just `\verb|q().|' to exit from
  the Erlang shell. This probably can take some time if Mnesia has not yet
  transfered and processed all data it needed from \term{first}.


\item Now run \ejabberd{} on \term{second} with almost the same config as
  on \term{first} (you probably do not need to duplicate `\verb|acl|'
  and `\verb|access|' options --- they will be taken from
  \term{first}, and \verb|mod_muc| and \verb|mod_irc| should be
  enabled only on one machine in the cluster).
\end{enumerate}

You can repeat these steps for other machines supposed to serve this
domain.

\section{Service Load-Balancing}
\subsection{Components Load-Balancing}
\label{componentlb}
\ind{component load-balancing}

\subsection{Domain Load-Balancing Algorithm}
\label{domainlb}
\ind{options!domain\_balancing}

\ejabberd{} includes an algorithm to load balance the components that are plugged on an ejabberd cluster. It means that you can plug one or several instances of the same component on each ejabberd cluster and that the traffic will be automatically distributed.

The default distribution algorithm try to deliver to a local instance of a component. If several local instances are available, one instance is choosen randomly. If no instance is available locally, one instance is choosen randomly among the remote component instances.

If you need a different behaviour, you can change the load balancing behaviour with the option \option{domain\_balancing}. The syntax of the option is the following:

\begin{verbatim}
	{domain_balancing, "component.example.com", <balancing_criterium>}.                                   
\end{verbatim}

Several balancing criteria are available:
\begin{itemize}
\item \term{destination}: the full JID of the packet \term{to} attribute is used.
\item \term{source}: the full JID of the packet \term{from} attribute is used.
\item \term{bare\_destination}: the bare JID (without resource) of the packet \term{to} attribute is used.
\item \term{bare\_source}: the bare JID (without resource) of the packet \term{from} attribute is used.
\end{itemize}

If the value corresponding to the criterium is the same, the same component instance in the cluster will be used.

\subsection{Load-Balancing Buckets}
\label{lbbuckets}
\ind{options!domain\_balancing\_component\_number}

When there is a risk of failure for a given component, domain balancing can cause service trouble. If one component is failling the service will not work correctly unless the sessions are rebalanced.

In this case, it is best to limit the problem to the sessions handled by the failling component. This is what the \term{domain\_balancing\_component\_number} option does, making the load balancing algorithm not dynamic, but sticky on a fix number of component instances.

The syntax is the following:
\begin{verbatim}
    {domain_balancing_component_number, "component.example.com", N}
\end{verbatim}



% TODO
% See also the section about ejabberdctl!!!!
%\section{Backup and Restore}
%\label{backup}
%\ind{backup}

\chapter{Debugging}
\label{debugging}
\ind{debugging}

\section{Watchdog alerts}
\label{watchdog}
\ind{debugging!watchdog}

ejabberd includes a watchdog mechanism to notify admins in realtime
through XMPP when any process consumes too much memory.

To enable the watchdog, add the \term{watchdog\_admins}
\ind{options!watchdog\_admins} option in the config file:

\begin{verbatim}
{watchdog_admins, [``admin@localhost'']}.
\end{verbatim}

\appendix{}
\chapter{Internationalization and Localization}
\label{i18nl10n}
\ind{xml:lang}\ind{internationalization}\ind{localization}\ind{i18n}\ind{l10n}

All built-in modules support the \texttt{xml:lang} attribute inside IQ queries.
Figure~\ref{fig:discorus}, for example, shows the reply to the following query:
\begin{verbatim}
  <iq id='5'
      to='example.org'
      type='get'
      xml:lang='ru'>
    <query xmlns='http://jabber.org/protocol/disco#items'/>
  </iq>
\end{verbatim}

\begin{figure}[htbp]
  \centering
  \insimg{discorus.png}
  \caption{Service Discovery when \texttt{xml:lang='ru'}}
  \label{fig:discorus}
\end{figure}

The web interface also supports the \verb|Accept-Language| HTTP header (compare
figure~\ref{fig:webadmmainru} with figure~\ref{fig:webadmmain})

\begin{figure}[htbp]
  \centering
  \insimg{webadmmainru.png}
  \caption{Top page from the web interface with HTTP header
    `Accept-Language: ru'}
  \label{fig:webadmmainru}
\end{figure}


%\section{Ultra Complex Example}
%\label{ultracomplexexample}
%TODO: a very big example covering the whole guide, with a good explanation before the example: different authenticaton mechanisms, transports, ACLs, multple virtual hosts, virtual host specific settings and general settings, modules,...

\chapter{Release Notes}
\label{releasenotes}
\ind{release notes}

Release notes are available from \footahref{}{ejabberd Home Page}

\chapter{Acknowledgements}
\label{acknowledgements}
Thanks to all people who contributed to this guide:
\begin{itemize}
\item Alexey Shchepin (\ahrefurl{xmpp:aleksey@jabber.ru})
\item Badlop (\ahrefurl{xmpp:badlop@jabberes.org})
\item Evgeniy Khramtsov (\ahrefurl{xmpp:xram@jabber.ru})
\item Florian Zumbiehl (\ahrefurl{xmpp:florz@florz.de})
\item Michael Grigutsch (\ahrefurl{xmpp:migri@jabber.i-pobox.net})
\item Mickael Remond (\ahrefurl{xmpp:mremond@erlang-projects.org})
\item Sander Devrieze (\ahrefurl{xmpp:sander@devrieze.dyndns.org})
\item Sergei Golovan (\ahrefurl{xmpp:sgolovan@nes.ru})
\item Vsevolod Pelipas (\ahrefurl{xmpp:vsevoload@jabber.ru})
\end{itemize}


\chapter{Copyright Information}
\label{copyright}

Ejabberd Installation and Operation Guide.\\
Copyright \copyright{} 2003 --- 2007 Process-one

This document is free software; you can redistribute it and/or
modify it under the terms of the GNU General Public License
as published by the Free Software Foundation; either version 2
of the License, or (at your option) any later version.

This document is distributed in the hope that it will be useful,
but WITHOUT ANY WARRANTY; without even the implied warranty of
MERCHANTABILITY or FITNESS FOR A PARTICULAR PURPOSE. See the
GNU General Public License for more details.

You should have received a copy of the GNU General Public License along with
this document; if not, write to the Free Software Foundation, Inc., 51 Franklin
Street, Fifth Floor, Boston, MA 02110-1301, USA.

%TODO: a glossary describing common terms
%\section{Glossary}}
%\label{glossary}
%\ind{glossary}

%\begin{description}
%\titem{c2s} 
%\titem{s2s}
%\titem{STARTTLS}
%\titem{XEP} (\XMPP{} Extension Protocol)
%\titem{Resource}
%\titem{Roster}
%\titem{Transport}
%\titem{JID} (\Jabber{} ID) <Wikipedia>
%\titem{JUD} (\Jabber{} User Directory)
%\titem{vCard} <Wikipedia>
%\titem{Publish-Subscribe}
%\titem{Namespace}
%\titem{Erlang} <Wikipedia>
%\titem{Fault-tolerant}
%\titem{Distributed} <Wikipedia>
%\titem{Node} <Wikipedia>
%\titem{Tuple} <Wikipedia>
%\titem{Regular Expression}
%\titem{ACL} (Access Control List) <Wikipedia>
%\titem{IPv6} <Wikipedia>
%\titem{Jabber}
%\titem{LDAP} (Lightweight Directory Access Protocol) <Wikipedia>
%\titem{ODBC} (Open Database Connectivity) <Wikipedia>
%\titem{Virtual Hosting} <Wikipedia>

%\end{description}



% Remove the index from the HTML version to save size and bandwith.
\begin{latexonly}
\printindex
\end{latexonly}

\end{document}
